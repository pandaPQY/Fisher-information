\begin{section}{Reconstruction Algorithm}
  \label{sec:reconstruction}
  In this section, we briefly review the MM algorithm; for a more
  complete description we refer the reader to \cite{bib:ZhuH2016}.  
  The aim of the MM algorithm is to estimate the displacement of particles in 
  Lagrangian coordiantes from their final Eulerian position only. The
  general principle is to relate a particle's Eulerian coordinates, $x^i$ to
  a curvilinear system, $\xi^\mu$, in which the number of particles
  per grid cell is approximately constant.  These coordinates are
  related via the so-called deformation, which we assume to be a pure
  gradient:
  \begin{align}
    x^i = \xi^\mu \delta^i_\mu + \frac{\partial \phi}{\partial
    \xi^\mu}\delta^{i\mu}
  \end{align}
  and $\phi$ is called the deformation potential.  Numerially, we 
  can solve iteratively for
  the deformation potential via the diffusion equation, 
  \begin{align}
    \label{eq:li_elip}
    \partial _\mu (\rho \sqrt{g} e^\mu _i \delta^{i\nu}
    \partial_\nu \dot{\phi})=\Delta \rho,
  \end{align}
  where $e^i_\mu = \partial x^i / \partial \xi ^ \mu$ is the
  coordinate transformation matrix,
  $\sqrt{g} \equiv \mathrm{det}\left| e^i_\mu\right|$ is the volume
  element and $\Delta \rho = \bar{\rho}-\rho \sqrt{g}$. Detailed discription 
  of the analytical formulation were presented in \cite{bib:Pen1995,bib:Pen1998}.
  Eq.~\ref{eq:li_elip} can be solved through the use of the multigrid
  algorithm described in \cite{bib:Pen1995,bib:Pen1998,bib:ZhuH2016}.
  The displacement field is then given by
  \begin{align}
   \label{eq:disp}
   \bs{\Psi}(\bs{\xi})=\nabla \phi(\bs{\xi}),
  \end{align}
  and the reconstructed density field is given by
  \begin{align}
   \label{eq:delta}
   \delta_R(\bs{\xi})=- \nabla \cdot \bs{\Psi}(\bs{\xi}) = \nabla^2 \phi(\bs{\xi}). 
  \end{align}

\end{section}

% \tcb{Detailed description of MM algorithm can be found in
% \cite{bib:ZhuH2016}. Here we quickly review it for completeness.}
% The aim of the reconstruction is to estimate the Lagrangian
% position, or the displacement, of particles from their final
% position only.  Due to the highly nonlinear process at the late
% stage, it is hard to fully discrible the displacement field from the
% final condition.  However, since the result in \cite{bib:Yu2016}
% showed that the $E$-mode displacement field has a strong correlation
% with the linear density field on large scale ($r > 0.5$ when
% $k \lesssim 2 h/\mathrm{Mpc}$), estimating the $E$-mode displacement
% is expected to recover a large amount of information.  It can be
% done by applying \textit{Moving-Mesh} (MM) algorithm discribed in
% \cite{bib:ZhuH2016}, which is originated from \textit{Adapting
% Particle-Mesh} (APM) algorithm \cite{bib:Pen1995,bib:Pen1998}.

% The basic idea is to build a Particle-Mesh scheme on a curvilinear
% coordinate system, $\bm{\xi}=\left(\xi_1,\xi_2,\xi_3\right)$, in
% which number of particles per grid cell is set approximately
% constant.  The displacement of particles in each grid cell is then
% approximately discribed by the deformation of the curvilinear grid
% on the Eulidean coordinate $\bm{x}(\bm{\xi},t)$.  Assuming that the
% deformation is a pure gradient, the physicsl position of particles
% on Eular coordinate is given as
% \begin{align}
%   x^i=\xi ^\mu \delta ^i _\mu + \Delta x^i,
% \end{align}
% where
% \begin{align}
%   \label{eq:disp}
%   \Delta x^i=\frac{\partial \phi}{\partial \xi ^ \nu}\delta ^{i \nu}
%     %   .
%  \end{align}
%  We use the convention the same as in \cite{bib:Pen1995}, Latin
%  indices denoting Cartesian coordinate, while Greek indices denoting
%  the curvilinear grid coordinate.  $\phi$ is called the deformation
%  potential, and $\Delta x^i$ the lattice displacement.  The
%  coordinate transfromation matrix
%  $e^i_\mu = \partial x^i / \partial \xi ^ \mu$ is guarantee to be
%  positive definite so that the volume element
%  $\sqrt{g} \equiv \mathrm{det}\left| e^i_\mu\right|$ is always
%  positive.  This choice of the deformation can minimize the
%  cell-crossing.

%  To solve the deformation potential $\phi$, consider the continuity
%  equation in curvilinear coordiante,
%  \begin{align}
%    \label{eq:continue_eq}
%    \frac{\partial (\sqrt{g} \rho) }{\partial t}+
%    \partial_\mu \left[\rho \sqrt{g} e^\mu _i \left(v^i -
%        \Delta \dot{x}^i \right) \right] =0
%  \end{align}
%  $\Delta \dot{x}^i=\delta ^{i\nu}\partial _\nu \dot{\phi}$ is chosen
%  such that the first term in Eq.\ref{eq:continue_eq} is zero,
%  resulting in a constant mass per volume element.  The velocity
%  field divergence is replaced by the deviation density field
%  $\Delta \rho = \bar{\rho}-\rho \sqrt{g}$, which ideally should be
%  zero.  Then the deformation potential is described via the elliptic
%  equation,
%  \begin{align}
%    \label{eq:li_elip}
%    \partial _\mu (\rho \sqrt{g} e^\mu _i \delta^{i\nu}
%    \partial_\nu \Delta \phi)=\Delta \rho.
%  \end{align}
%  The Eq.\ref{eq:li_elip} can be solved using multigrid algorithm
%  described in \cite{bib:Pen1995,bib:Pen1998} (see also
%  \cite{bib:ZhuH2016} for brief discription).  The deformation
%  $\Delta x^i$ is closer to the displacement of particles when higher
%  resolution is used to decribed the density field so that less
%  particles are contained in each grid cell.
