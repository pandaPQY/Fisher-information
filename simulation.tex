\begin{section}{Implementation and Power Spectra}
  \label{sec:simulation}
  We use the \textsc{CUBEP$^3$M} code \cite{bib:Harnois2013} to run
  140 simulations with a box size of 600 Mpc/h and $512^3$ particles.
  The initial conditions are computed using the transfer function
  given by CAMB \cite{bib:Lewis2000} and then propagating the power
  back to $z=100$ with a linear growth factor.  The Zel'dovich
  approximation is used to calculate the displacement and velocity
  fields of the particles.  For these simulations, we use cosmological
  parameters $\Omega_M=0.321$, $\Omega_{\Lambda}=1.0-\Omega_m$,
  $h=0.67$, $\sigma_8=0.83$, and $n_s=0.96$.  Different random seeds
  are used to produce the initial conditions for different simulations
  so that they are independent of each other.

  \begin{figure}[h]
    \centering
    \includegraphics[width=0.45\textwidth]{fig1.pdf}
    \caption{ The 2-D projection of one layer of the deformed grid of a sample
      $N$-body simulation is shown as curved white lines.  The
      density fluctuation on the grid, $\delta\rho/\bar{\rho}$, is shown
      underneath.}
    \label{fig:simandrec}
 \end{figure}

  We use Voronoi tessellation method to estimate the density
  contrast $\delta_S=\delta\rho/\rho-1$ from the particles, and then
  apply the MM reconstruction to these fields with a resolution of
  $512^3$ cells.  The reconstruction code solves the dispacement potentials 
  iteratively until the root mean square (rms) of the results drop (from $\sim 7.5$) 
  to 0.20. For different simulation samples, different time steps are taken to get 
  the results of the same rms. We finally pick 130 simulation samples to do the remaining calculation, 
  each among which gives a final rms of 0.20
  through running the reconstruction code for no more than 2000 time steps. 
  A 2-D projection of one layer of the deformed grids and the
  original density field on the grids are given in Fig.~\ref{fig:simandrec}.  As
  expected, there is no grid crossing after reconstruction.
 
 The cross power spectrum, $P_{ab}(k)$, is defined as
 \begin{align}
   \langle \delta_a(\bm{k})\delta_b(\bm{k'}) \rangle =
   (2\pi)^3 P_{ab}(k) \delta_{3D}(\bm{k}-\bm{k'}),
 \end{align}
 where $\delta_{a}$ and $\delta_{b}$ are any density contrasts and
 $\delta_{3D}$ is the three-dimensional Dirac delta funciton. We typically consider instead
 the dimensionless power spectrum, $\Delta_{ab}^2(k)$, defined as
 \begin{align}
   \Delta_{ab}^2(k) \equiv \frac{k^3 P_{ab}(k)}{2\pi ^2}.
 \end{align}
 In the left panel of Fig.~\ref{fig:cp}, we show the matter auto power
 spectrum ($a=b$) of linear theory density fields ($\delta_L$), from
 the simulation results ($\delta_S$) and after reconstruction
 ($\delta_R=-\nabla^2\phi$).  For the simulation
 results, we use the average value of all 130 simulations and show
 $1\sigma$ variances as error bars.  To determine the correlation
 between fields, we compute the cross correlation coefficient
 $r_{ab}(k) = P_{ab}/\sqrt{P_{aa}P_{bb}}$.  In the right panel of
 Fig.~\ref{fig:cp}, we show $r_{SL}$ and $r_{RL}$.  We see that the
 reconstructed field is much more highly correlated with the linear
 field than the simulation field is.  For comparison, we also plot the 
 correlation coefficient of $\delta_E$ 
 and $\delta_L$ as the result in \cite{bib:Yu2016}, 
 where $\delta_E(\bs{q})= - \nabla_q \cdot \bs{\Psi}(\bs{q})$ is the 
 negative divergence of the real non-linear displacement from simulaiton. 
 Ideally, MM algorithm aims at getting the cross correlation $r_{RL}$ approaching to $r_{EL}$. 
 Even though $r_{RL}$ is off the $r_{EL}$ curve in non-linear regime due to the fact that MM reconstruction 
 cannot recover the cell-crossing and vorticity on these scales, we find that on the scale $k\simeq 0.05$ 
 h/Mpc to $0.3$ h/Mpc, linear modes were recovered successfully.
 Specifically, the scale at which $r(k)=1/2$ drops from $k\simeq 0.2$ h/Mpc to
 $0.8$ h/Mpc after reconstruction.  In comparison with the results of \citet{bib:ZhuH2016},
 we find the correlation coefficient falls off at slightly lower
 wavenumbers, which we attribute to using less particles to run the simulations.

  \begin{figure*}
    \centering
    \includegraphics[width=0.5\textwidth]{fig2a.pdf}
    \includegraphics[width=0.49\textwidth]{fig2b.pdf}
    \caption{{\it Left.} The dimensionless power spectrum computed via
      linear theory (black), the mean value of 130 $N$-body
      simulations with $1\sigma$ error bars (blue), and reconstruction
      of the simulations (red).  {\it Right.} The cross correlation
      function $r_{SL}$ (blue) and
      $r_{RL}$ (red), and $r_{EL}$ (dash
      lines).}
    \label{fig:cp}
  \end{figure*}


\end{section}

