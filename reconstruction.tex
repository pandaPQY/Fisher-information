\begin{section}{Reconstruction Algorithm}
  \label{sec:reconstruction}
   The basic idea of the MM algorithm is to build a PM scheme on a curvilinear coordinate system, 
in which the number of particles per grid cell is set approximately constant. 
    Consider a numerical grid of curvilinear coordinates 
$\bm{\xi}=\left(\xi_1,\xi_2,\xi_3\right)$. In order to determine the physical position 
of each grid point, one needs to specify the Euclidean coordinate $\bm{x}(\bm{\xi},t)$ 
as a function of grid position. In the Euclidean coordinate, the flat metric is the Kronecker 
delta function $\delta_{ij}$, while the curvilinear metric is given by
\begin{align}
    g_{\mu\nu}=\frac{\partial x^i}{\partial \xi ^\mu} \frac{\partial x^j}{\partial \xi ^\nu}\delta_{ij}.
\end{align}
    We use the convention that Latin indices denote Cartesian coordinate, 
while Greek indices denote the curvilinear grid coordinate.
    In principle, there are many different methods to connect the Cartesian 
coordinate and curvilinear coordinate of each grid cell. In the APM method, the 
connection is described by an irrotational deformation,
\begin{align}
    x^i=\xi ^\mu \delta ^i _\mu + \Delta x^i,
\end{align}
where
\begin{align}
 \label{eq:disp}
    \Delta x^i=\frac{\partial \phi}{\partial \xi ^ \nu}\delta ^{i \nu} .
\end{align}
    This choice of the deformation can minimize mesh distortion and twisting. $\phi$ is called 
the deformation potential, and $\Delta x^i$ the lattice displacement. The deformation potential 
can be given in terms of the continuity equation in curvilinear coordinate,
\begin{align}
 \label{eq:continue_eq}
    \frac{\partial (\sqrt{g} \rho) }{\partial t}+\partial_\mu \left[\rho \sqrt{g} e^\mu _i \left(v^i - \Delta \dot{x}^i \right) \right] =0
\end{align}
where $\sqrt{g} \equiv \mathrm{det}\left| e^i_\mu\right|$ is the volume element 
and thus $\sqrt{g} \rho$ represents the particle mass in the volume element under the curvilinear 
coordinate system. The triad is given by $e^i_\mu = \partial x^i / \partial \xi ^ \mu$. 
$\Delta \dot{x}^i=\delta ^{i\nu}\partial _\nu \dot{\phi}$ is chosen such that the first 
term in Eq.\ref{eq:continue_eq} is zero, resulting in a constant mass per volume 
element. The velocity field divergence is replaced by the deviation density field 
$\Delta \rho = \bar{\rho}-\rho \sqrt{g}$, which ideally should be zero. Then the 
deformation potential is described via the elliptic equation,
\begin{align}
 \label{eq:li_elip}
    \partial _\mu (\rho \sqrt{g} e^\mu _i \delta^{i\nu}\partial_\nu \Delta \phi)=\Delta \rho.
\end{align}
The Eq.\ref{eq:li_elip} can be solved using multigrid algorithm described in Ref. 
\cite{bib:Pen1995,bib:Pen1998}. The displacement is then given by the gradiant of the deformation 
potential as in Eq.\ref{eq:disp}. We run the MM reconstruction code on 136 nonlinear density fields from simulation 
in a resolution of ng $=128$ per dimension. The multigrid algorithm is iterated for 1000 times then the root mean square  
is decreased from $\sim$ 4.5 to $\sim$ 0.2. 
A 2D projection of one layer of the deformed grids and the 
original density field on the grid are given in Fig.\ref{fig:simandrec}. 
As expected, there is no grid crossing after reconstruction.
%
\end{section}

