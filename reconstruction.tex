\begin{section}{Reconstruction Algorithm}
  \label{sec:Reconstruction}
    We use the algorithm and numerical method called Adaptive Particle-mesh (APM), discribed in citebib:...,bib:... to reconstruct the density field. The basic idea is to build a PM scheme on a curvilinear coordinate system, in which the number of the particles per grid cell is set approximately constant. 
    Consider a numerical grid of curvilinear coordinates $\bm{\xi}=\left(\xi_1,\xi_2,\xi_3\right)$. In order to determine the physical position of each grid point, one needs to specify the Euclidean coordinate $\bm{x}(\bm{\xi},t)$ as a function of grid position. In the Euclidean coordinate, the flat metric is Kronecker delta function $\delta_{ij}$, while the curvilinear metric is given by
\begin{align}
    g_{\mu\nu}=\frac{\partial x^i}{\partial \xi ^\mu} \frac{\partial x^j}{\partial \xi ^\nu}\delta_{ij}.
\end{align}
    We use the convention that Latin indices denote Cartesian coordinate, while Greek indices denote the curvilinear grid coordinate.
    In principle, there are many different methods to connect the Cartesian coordiante and curvilinear coordinate of each grid cell. In APM method, the connetction is described by an irrotational deformation,
\begin{align}
    x^i=\xi ^\mu \delta ^i _\mu + \Delta x^i,
\end{align}
where
\begin{align}
 \label{eq:disp}
    \Delta x^i=\frac{\partial \phi}{\partial \xi ^ \nu}\delta ^i _\nu .
\end{align}
    This choice of the deformation can minimize mesh distortion and twisting. $\phi$ is called the deformation potential, and $\Delta x^i$ the lattice displacement. The deformation potential can be given in terms of the continuity equation in curvilinear coordinate,
\begin{align}
 \label{eq:continue_eq}
    \frac{\partial \sqrt{g} \rho }{\partial t}+\partial_\mu \left[\rho \sqrt{g} e^\mu _i \left(v^i - \Delta \dot{x}^i \right) \right] =0
\end{align}
where $\sqrt{g} \equiv (\partial x^i / \partial \xi ^ \alpha)$ is the volume element and $e^i _\mu = \partial \xi ^\mu / \partial x^i$ is the triad. $\Delta \dot{x}=\delta ^{i\nu}\partial _\nu dot{\phi}$ is chosen such that the first term in equation \ref{eq:continue_eq} is zero, resulting in a constant mass per volume element. And the velocity field divergence is replaced by the deviation density field $\Delta \rho = \bar{\rho}-\rho \sqrt{g}$, which ideally should be zero. Then the deformation potential is described in the elliptic equation,
\begin{align}
 \label{eq:li_elip}
    \partial _\mu (\rho \sqrt{g} e^\mu _i \delta^{i\nu}\partial_\nu \Delta \phi)=\Delta \rho
\end{align}
The equation \ref{eq:li_elip} can be solved using multigrid algorithm described in Ref. .... Then the displacement is given by the gradiant of the deformation potential as in \ref{eq:disp}. One layer of the deformed grids and the original density field of that layer is given in Fig. \ref{fig:reconstrucion}. As expeted, there's no grid crossing after reconstruction.
\begin{figure}[htbp]
 \begin{center}
  \includegraphics[width=0.5\textwidth]{reconstruction.eps}
   \caption{The density field and the deformed grids of a random selected layer of a random selected density field from 139 simulation.}
  \label{fig:reconstruction}
 \end{center}
\end{figure}

     
%%  \begin{subsection}{Sound speeds}
%    In solving Eq. \ref{eq:fluid_eqn}, we assumed a constant sound
%    speed despite the fact that $\Pi = -c_s^2 k^2 \delta$ does not
%    enforce $c_s$ to be constant.  Since we have solutions for $\Pi$
%    and $\delta$ in Eq. \ref{eq:vlasov_str} and \ref{eq:vlasov_den},
%    we can compute the exact sound speed as a function of wavenumber:
%    \begin{align}
%      c_s^2  = \frac{1}{\beta^2}\frac{ \int_{-\infty}^{s} ds'
%      a^2(-k^2\phi)(s-s') \langle u^2 j_0(ku(s-s')/\beta) \rangle_0 }{ \int_{-\infty}^{s} ds'
%      a^2(-k^2\phi)(s-s') \langle j_0(ku(s-s')/\beta) \rangle_0 }.
%    \end{align}
%    We show this as solid lines in Fig. \ref{fig:vlasov_cs}.
%    \begin{figure}[htbp]
%      \begin{center}
%        \includegraphics[width=0.5\textwidth]{./vlasov_cs_mass.eps}
%        \caption{Sound speeds computed from the linearized Vlasov
%          equation.  Solid lines are computed with respect to the
%          stress , $\Pi$, with darker lines indicating heavier
%          neutrino mass.  The dashed and dash-dotted lines are
%          computed with respect to the pressure $k^2P$ and the
%          anisotropic stress $\tau$ for $m_\nu = 50$ \mev.  Horizontal
%          dotted lines are pre-computed asymptotic behaviours which
%          are independent of both neutrino mass and time.
%          $\beta=m/(k_BT_\nu c)$.}
%        \label{fig:vlasov_cs}
%      \end{center}
%    \end{figure}
%    We see that neutrinos are approximately bimodal with constant
%    sound speed at large and small $k$.  Both these values are
%    computable.  For $k\rightarrow 0$, $j_0(ku(s-s')/\beta) \simeq 1$.
%    This means the velocity integral becomes separable from the $s$
%    integral and we find
%    \begin{align}
%      \label{eq:cs_lowk}
%      -\frac{\beta^2}{k^2}\Pi &\simeq 
%                                \int ds' a^2(-k^2\phi)(s-s')
%                                \sigma^2 \nonumber \\
%      \delta &\simeq \int ds' a^2(-k^2\phi)(s-s')  \\
%      \therefore  c_s^2 &=\left( \frac{\sigma}{\beta} \right)^2 \nonumber
%    \end{align}
%    where $\sigma$ is the velocity dispersion
%    $\sigma^2 = \langle u^2 \rangle_0 \simeq 12.94$.  For large $k$,
%    the sinusoids in the velocity integral oscillate rapidly and add
%    to zero unless $s'\simeq s$
%    \cite{bib:Ringwald2004,bib:AliHaimoud2012}.  Under this
%    assumption, $(a\delta_m)(s') \simeq (a\delta_m)(s)$ and can be
%    factored out of the integral yielding:
%    \begin{align}
%      \label{eq:cs_highk}
%      -\frac{\beta^2}{k^2}\Pi &\simeq a^2(-k^2\phi) \left(\frac{\beta}{k}\right)^2 \nonumber \\
%      \delta &\simeq a^2(-k^2\phi) \left(\frac{\beta}{k}\right)^2
%               \Sigma^{-2}  \\
%      \therefore c_s^2& =  \left(\frac{\Sigma}{\beta}\right)^2 \nonumber
%    \end{align}
%    where one must be particularly careful in changing the order of
%    integration and $\Sigma$ is an ``inverse dispersion''
%    $\Sigma^{-2} = \langle u^{-2} \rangle_0 \simeq 0.38$.  This second
%    sound speed is also the one that goes into defining the free
%    streaming wavenumber
%    $k_{fs} = \sqrt{ \frac{3}{2} \Omega_m a }
%    H_0\frac{\beta}{\Sigma}$.
%    A simple ``instantaneous'' approximation on small scales is easily
%    found by considering Eq. \ref{eq:fluid_eqn}: for large $k$,
%    $\delta_{ss} \ll c_s^2 k^2 \delta$ and so
%    $\delta \simeq \frac{-a^2k^2\phi}{c_s^2 k^2}$ or:
%    \begin{align}
%      \label{eq:delta_highk}
%      \delta \simeq \left(\frac{k_{fs}}{k} \right)^2 \delta_m
%    \end{align}
%    on small scales.  Equivalently, we can treat this as an equation
%    for the sound speed:
%    \begin{align}
%      \label{eq:cs_highk_den}
%      c_s^2 = \frac{ \frac{3}{2}H_0^2 \Omega_m a}{k^2} \frac{\delta_m}{\delta}.
%    \end{align}
%    Finally, we can divide the stress tensor into pressure, $P$, and
%    anisotropic stress,$\tau$, $\Pi = -k^2P + \tau$.  Equations for
%    these components are again derived in the Supplement:
%    \begin{align}
%      \beta^2 P &= \int_{-\infty}^s ds' a^2 (-k^2\phi) (s-s') \frac{1}{3} \langle u^2
%                  (j_0(ku(s-s')/\beta)+\nonumber \\ &\hspace{1cm}+2 j_1(ku(s-s')/\beta)/(ku(s-s')/\beta) \rangle_0 \label{eq:vlasov_pre} \\
%      -\frac{\beta^2}{k^2} \tau &= \int_{-\infty}^s ds' a^2 (-k^2\phi) (s-s') \frac{2}{3} \langle u^2
%                                  (j_0(ku(s-s')/\beta) \nonumber \\
%                &\hspace{1cm}-j_1(ku(s-s')/\beta)/(ku(s-s')/\beta) \rangle_0 \label{eq:vlasov_ani}
%    \end{align}
%    We can now repeat our small scale approximations for these two
%    components, e.g. $P/(\delta)$ and $\tau/(-k^2\delta)$.  For the
%    pressure on small scales, we expand the sinusoides to find
%    $\frac{1}{3} (j_0(x)+2j_1(x)/x) \simeq \frac{5}{9} x$ and so the
%    sound speed is $\sqrt{5/9}\sigma$.  This result was derived by
%    \cite{bib:Shoji2010}.  On small scales, repeating the above
%    derivation shows that $k^2 P\rightarrow \Pi$ and
%    $\tau\rightarrow 0$.  These approximations are shown as horizontal
%    lines in Fig. \ref{fig:vlasov_cs} and closely match the integrated
%    values (dashed and dash-dotted lines).
%  \end{subsection}
%
%  \begin{subsection}{Perturbed Distribution Function}
%    We now repeat the arguments used in computing the large and small
%    scale sound speed limits but for the distribution function
%    instead.  Assuming negligible initial conditions,
%    Eq. \ref{eq:vlasov_sln} can be written as:
%    \begin{align}
%      f(s,k,v,\mu) = \int_{-\infty}^s ds' a^2 i k \phi \mu
%      \frac{df^0}{dv} e^{-ikv\mu(s-s')} \nonumber
%    \end{align}
%    where $\mu=\vec{k}\cdot\vec{v}/(kv)$.  We can now integrate over
%    angles to find:
%    \begin{align}
%      \langle f(s,k,v) \rangle &= \frac{1}{2}\int_{-1}^1d\mu 
%                                 f(s,k,v,\mu) \nonumber \\
%                               &=\frac{df^0}{dv}\int_{-\infty}^{s} ds'
%                                 a^2 k \phi j_1(kv(s-s')). \nonumber
%    \end{align}
%    In the large scale limit, $kv(s-s')\ll1$ and
%    $j_1(kv(s-s')) \simeq kv(s-s')/3$.  Using this approximation and
%    substituting the density obtained in Eq. \ref{eq:cs_lowk} yields:
%    \begin{align}
%      \langle f \rangle = -\frac{1}{3} v \frac{df^0}{dv} \delta. \nonumber
%    \end{align}
%    In the small scale limit limit limit, we again use $s\simeq s'$
%    and find
%    \begin{align}
%      \langle f \rangle &= \frac{df^0}{dv} a^2 k \phi \int_{-\infty}^s
%                          ds' j_1(kv(s-s')) \nonumber \\
%                        &=\frac{1}{v}\frac{df^0}{dv} a^2 \phi
%                          \int_0^\infty j_1(x) dx \nonumber \\
%                        &=-\frac{1}{v}\frac{df^0}{dv}\frac{\Sigma^2}{\beta^2}
%                          \delta \nonumber
%    \end{align}
%    where we use the density in Eq. \ref{eq:cs_highk} instead.  This
%    result was also computed in \cite{bib:AliHaimoud2012} using a
%    different technique.  We note now that strictly speaking these are
%    not ``low-k'' and ``high-k'' limits, rather, they refer to limits
%    where $kv\ll $ or $ \gg (\Delta s)^{-1}$ for some timescale
%    $\Delta s$.  Nonetheless we will refer to the limits as such
%    throughout the paper.  Both the low- and high-k perturbations are
%    separable in position and velocity and so we can define the
%    velocity space perturbation as
%    $f^1(v) = \langle f \rangle (v,k,s)/\delta(k,s)$.  In terms of the
%    dimensionless velocity $u=\beta v$ we have:
%    \begin{align}
%      \label{eq:perturbed_f}
%      \bar{f}^1(u) &= \frac{\langle f \rangle (u,k,s)}{\delta(k,s)} \nonumber \\
%                  &=\frac{1}{u}\frac{e^u}{\left ( e^u +1 \right )} \bar{f}^0(u) \begin{cases} \frac{1}{3}u^2
%                    & ku(\Delta s)/\beta \ll 1 \\ \Sigma^2 & ku(\Delta s)/\beta \gg
%                    1. \end{cases}
%    \end{align}
%
%    \begin{figure}
%      \begin{center}
%        \includegraphics[width=0.5\textwidth]{./distributions.eps}
%        \caption{The unperturbed Fermi-Dirac distribution $\bar{f}^0(u)$ as a
%          function of $u=\beta v$ is shown as a solid curve.  The
%          first order perturbations $\bar{f}^1(u)$ are shown as dashed
%          (low-k limit) and dash-dotted (high k limit).  Note that we
%          include the $u^2$ part of $d^3u$ in the distributions.
%          $\beta=m/(k_BT_\nu c)$}
%        \label{fig:distributions}
%      \end{center}
%    \end{figure}
%
%    We plot $\bar{f}^1(u)$ in Fig. \ref{fig:distributions} and compare
%    it to $\bar{f}^0(u)$.  We see that the
%    low-k limit tends to shift neutrinos to higher velocities;
%    presumably due to the gravitational accelerations.  This is
%    qualitatively consistent with the velocity distributions seen in
%    simulations, e.g. Fig. 4 and 13 of
%    \cite{bib:VillaescusaNavarro2012}.  On the other hand, the high-k
%    limit favours low velocity neutrinos and, to our knowledge, has
%    not previously been seen.  We have also been unable to find
%    particles distributed this way in simple tests of our own
%    simulations.
%
%    If neutrinos were distributed according to $\bar{f}^1(u)$ rather
%    than $\bar{f}^0(u)$, the asymptotic sound speeds would change.
%    For the $kv(\Delta s) \ll 1$ limit, the asymptotic values would be
%    $\sqrt{\langle u^2 \rangle_1} = \sqrt{5/3}\sigma$ at low-k and
%    $1/\sqrt{\langle u^{-2} \rangle_1}=\sqrt{3}\Sigma$ at high k.  For
%    $kv(\Delta s) \gg 1$, the low-k asymptote becomes
%    $\sqrt{\langle u^2 \rangle_1} = \sqrt{3}\Sigma$; however, the
%    high-k asymptote goes to zero.  This is due to integrating from
%    $u=0$, which clearly violates $ku(\Delta s)/\beta \gg 1$ regardless
%    of k (the reverse case, for low-k, is less of a problem as
%    $\bar{f}^0$ is truncating $u\rightarrow\infty$ and we can also
%    simply consider $k=0$).  Hence, the sound speed need not necessarily be
%    zero as the approximation technique is somewhat inapplicable.
%    Nonetheless, the high-k perturbation is more sensitive
%    to low velocity neutrinos and therefore the asymptotic sound
%    speed should decrease when including higher perturbations.
%
%  \end{subsection}
%
%  \begin{subsection}{Simulation Sound Speed}
%    \label{ssec:simcs}
%
%    Eq. \ref{eq:cs_highk_den} depends only on the total matter density
%    field and the neutrino density field.  We can therefore use our
%    simulation power spectra to estimate the sound speed with the
%    approximation $\delta(k) = \sqrt{\Delta^2(k)}$.  We show the
%    results in Fig. \ref{fig:data_cs}.  We find that this estimate of
%    the sound speed is significantly lower than the linear theory
%    prediction of Eq. \ref{eq:cs_highk} with values $\beta c_s \sim 1$
%    rather than $\sim \Sigma$.  There is also now significant mass
%    dependence of the asymptotic value, consistent with our
%    expectation that heavier neutrinos should behave less linearly,
%    and more like CDM (which has no sound speed).
%
%    \begin{figure}[htbp]
%      \begin{center}
%        \includegraphics[width=0.5\textwidth]{./data_cs.eps}
%        \caption{Estimates of the simulation sound speed for different
%          neutrino masses.  Dots are estimated using the instantaneous
%          approximation (Eq. \ref{eq:cs_highk_den}) and are valid only
%          at high k.  Lines show asymptotic values (note that
%          $\varsigma$ is calibrated to the instantaneous approximation
%          shown).  $\beta=m/(k_BT_\nu c)$}
%        \label{fig:data_cs}
%      \end{center}
%    \end{figure}
%    With this behaviour we can now interpret the discrepency between
%    linear response and N-body.  The k-dependence of the sound speed
%    is proportional to $\delta_m/(k^2\delta)$.  By definition,
%    $\delta_m$ is perfectly linear in $\delta_m$.  Therefore, in order
%    to drive the sound speed to lower values $\delta$ must be larger.
%    This can only occur if, on small scales, non-linearities affect
%    $\delta$ much more than they do $\Pi$.  This makes sense as the
%    $\Pi$ is weighted by $v^2$ as compared to $v^0$ for $\delta$.
%    Hence, we expect Eq. \ref{eq:vlasov_str} to be more accurate than
%    Eq. \ref{eq:vlasov_den} as high velocity neutrinos behave more
%    linearly.  Furthermore, a decrease in $c_s$ causes the density to
%    grow more non-linearly, inducing feedback to continue decreasing
%    $c_s$.  Since low mass neutrinos are more linear in $\delta$, they
%    are less affected by this instability and so their sound speed is
%    closer to the linear theory asymptotic value, $\Sigma$.
%    
%  \end{subsection}
%
%  \begin{subsection}{Neutrino Power Spectrum}
%    \label{ssec:nupower}
%    We show the neutrino power spectrum for $m_\nu = 400$ \mev{} in the
%    top panel of Fig. \ref{fig:power}.  Black points correspond to the
%    N-body results.  Black dashed lines are linear response solutions,
%    integrated against linear theory (lower curve) or with a
%    non-linear correction:
%    $\delta_m \rightarrow \delta_m \sqrt{P_{NL}/P_{L}}$ (upper curve).
%    The dashed grey curve is the adiabatic approximation:
%    $P_\nu = (T_\nu/T_m)^2P_{NL}$.  We see that it is a reasonably
%    good fit to linear response.  However, neither linear response or
%    adiabatic solutions reproduce N-body results.  This is despite the
%    fact that the usual criterion for non-linearity, $\Delta^2(k) >
%    1$, is not met on any scale.  On the other hand,
%    the grey curve shows the solution corresponding to
%    Eq. \ref{eq:fluid_sln}, with a sound speed measured from
%    Fig. \ref{fig:data_cs}, and agrees very well with N-body on small
%    scales.  Finally, we show asymptotic behaviour
%    (e.g. Eq. \ref{eq:delta_highk}) for linear and non-linear
%    potentials as dotted lines.
%
%    \begin{figure}[htbp]
%      \begin{center}
%        \includegraphics[width=0.5\textwidth]{./power_mass.eps}
%        \caption{{\it Top Panel} Neutrino power spectrum at $z=0$ for
%          $m_\nu=400$ \mev.  Dashed black lines indicate linear response
%          to linear (lower curve) and \hfit{} (higher curve)
%          potentials.  The dashed grey line is the adiabatic
%          approximation.  The dots are from our N-body simulations.
%          The solid grey curve corresponds to the sound speed solution
%          chosen to match the high k behaviour.  Dotted lines are
%          asymptotic behaviours $\propto \delta_m/k^2$.  Finally, the
%          solid black line is our model. {\it Bottom Panel} Residuals
%          between model and N-body for different neutrino masses.}
%        \label{fig:power}
%      \end{center}
%    \end{figure}
%
%    We now present a simple model relating the neutrino power
%    spectrum, $\Delta^2_\nu(k,z,m_\nu)$ to the matter power spectrum,
%    $\Delta^2_m(k,z,m_\nu)$:
%    \begin{align}
%      \label{eq:pmodel}
%      \Delta^2_\nu = \Delta^2_m\left[\frac{T_\nu}{T_m} +
%      \left( \frac{k_\beta}{k}\right)^2\left( \frac{1}{\varsigma^2}-\frac{1}{\Sigma^2}\right)W(k/k_\varsigma) \right]^2
%    \end{align}
%    where $T_i$ are linear transfer functions computed via a Boltzmann
%    code such as \class{},
%    $k_\beta(z) = \Sigma k_{fs} = \sqrt{ \frac{3}{2} \Omega_m a } H_0
%    \beta$
%    is a typical fluid free-streaming scale neglecting the impact of
%    the Fermi-Dirac distribution,
%    $\beta c_s = \varsigma=\varsigma(z,m_\nu)$ is the best fitting
%    sound speed at high-k, and $W(k/k_\varsigma)$ is a high pass
%    filter that truncates the high-k accoustic behaviour.  
%    
%    We now explain each portion of the model.  The factor of
%    $T_\nu/T_m$ corresponds to linear behaviour under adiabatic
%    initial conditions (which, as previously noted, is quite close to
%    linear response when given the non-linear \hfit{} potential).  The
%    second term is the calibrated asymptotic behaviour at high-k, e.g.
%    $\delta_\nu \propto \delta_m/k^2$ after subtracting out the linear
%    behaviour (with $-1/\Sigma^2$).  We compute
%    $\varsigma = \beta c_s$ at redshift $z=0$ by averaging the last
%    three points in Fig. \ref{fig:data_cs}\footnote{For $m_\nu=100$
%      \mev{} the last point takes a sudden dip so we neglect it and
%      average the three points before that.  For $m_\nu=50$ \mev{} we
%      average three points in the same k-region as the other masses
%      and neglect one point that seems spuriously high.}.  We know it
%    must depend on time as it should go to its linear value, $\Sigma$,
%    at high redshift.  We compute $\varsigma$ for a few redshifts and
%    show the results in the bottom subpanel of
%    Fig. \ref{fig:redshift_cs}.
%    \begin{figure}[htbp]
%      \begin{center}
%        \includegraphics[width=0.5\textwidth]{./redshift_cs_mass.eps}
%        \caption{{\it Top Panel} N-body (dots), model (solid) and
%          adiabatic (dotted) power spectra at redshifts
%          $z=2.0, 1.0, 0.5,$ and $0.0$ for $m_\nu=200$ \mev.  {\it
%            Bottom Panel} Redshift dependence of
%          $\varsigma = \beta c_s$ evaluated at high k using the
%          instantaneous approximation (Eq. \ref{eq:cs_highk_den}) for
%          a variety of redshifts and masses.  At redshifts $z>2$ we
%          are unable to resolve the value.  Solid lines are for the
%          model given in Eq. \ref{eq:redshiftvar}.}
%        \label{fig:redshift_cs}
%      \end{center}
%    \end{figure}
%    We model the behaviour as:
%    \begin{align}
%      \label{eq:redshiftvar}
%      \varsigma(z) = \varsigma(0) + (\Sigma - \varsigma(0))Y(z)
%    \end{align}
%    where $Y(z)$ goes from $0$ at low redshift to $1$ at high
%    redshift.  We find $Y(z)=1-e^{-z/2}$ works reasonably well and is
%    shown in the bottom panel of Fig. \ref{fig:redshift_cs}.
%    Unfortunately we are unable to resolve the power spectra (and
%    hence $\varsigma$) at high k for neutrinos below $200$ \mev{} at
%    higher redshifts.  Finally, we find the following form for $W$
%    provides a good fit:
%    \begin{align}
%      \label{eq:hpfilter}
%      W(x) = \frac{1}{1+x^{-n}}
%    \end{align}
%    where $n$ must be $> 2$ so that at small scales we recover linear
%    behaviour\footnote{We note that filters of the form
%      $1/(1+(k/k_{fs})^2)$ have been shown to describe the neutrino
%      density contrast quite well in
%      \cite{bib:Ringwald2004,bib:AliHaimoud2012} - here we wish to
%      model from high-k to low-k so the exponent is negative.}.  We
%    find $n=2.25$ allows for good fits to all neutrino masses.  
%
%    Thus, our model has one
%    parameter that can be calibrated from simulations,
%    $\varsigma(m_\nu, z=0)$, one fitted parameter
%    $k_\varsigma(m_\nu)$, and two functions $W(k)$ and $Y(z)$, the
%    former depending on $n=2.25$.  We fit between
%    $0.2 < k/(\hmpc) < 9 $ (the lower bound is to avoid low-k
%    variance) and tabulate these values at $z=0$ in Table
%    \ref{tab:varsigma}.
%    \begin{table}[h]
%      \begin{tabular}{l | c | c | c | c}
%        \hline
%        Mass (\mev) & 50 & 100 & 200 & 400 \\\hline
%        $\varsigma=\beta c_s$ & 1.30 & 1.10 & 0.89 & 0.75 \\ %\hline
%        $k_{\varsigma} (\hmpc)$ & 1.11 & 0.92 & 1.20 & 1.65\\\hline
%        \end {tabular}
%        \caption{Parameters used in modeling the neutrino power
%          spectrum.  $\varsigma$ is the dimensionless sound speed calibrated from
%          high-k measurements in N-body simulations.  $k_\varsigma$ is a
%          best fit parameter.}
%      \label{tab:varsigma}
%    \end{table}
%    This model is shown as a solid black curve in Fig. \ref{fig:power}
%    and residuals for all neutrino masses are shown in the bottom
%    subpanel.  We see that in regions $0.1 < k/(\hmpc) < 10$ our model
%    is accurate to $~10\%$.  Since we do not consider time dependence
%    of $k_\varsigma$ (or $n$), at higher redshifts our model does not
%    describe the simulated power spectra as well.  For instance, for
%    $m_\nu = 400$ \mev{} there is over $ 50\%$ difference at $z=0.5$.
%    Nonetheless, this is still much better than linear response, as seen in
%    the top panel of Fig. \ref{fig:redshift_cs} where we show power
%    spectra at various redshifts along with the adiabatic
%    approximation for $m_\nu = 200$ \mev.
%
%  \end{subsection}

\end{section}

