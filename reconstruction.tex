\begin{section}{Reconstruction Algorithm}
  \label{sec:reconstruction}
   The basic idea of MM algorithm is to build a PM scheme on a curvilinear coordinate system, 
in which the number of the particles per grid cell is set approximately constant. 
    Consider a numerical grid of curvilinear coordinates 
$\bm{\xi}=\left(\xi_1,\xi_2,\xi_3\right)$. In order to determine the physical position 
of each grid point, one needs to specify the Euclidean coordinate $\bm{x}(\bm{\xi},t)$ 
as a function of grid position. In the Euclidean coordinate, the flat metric is Kronecker 
delta function $\delta_{ij}$, while the curvilinear metric is given by
\begin{align}
    g_{\mu\nu}=\frac{\partial x^i}{\partial \xi ^\mu} \frac{\partial x^j}{\partial \xi ^\nu}\delta_{ij}.
\end{align}
    We use the convention that Latin indices denote Cartesian coordinate, 
while Greek indices denote the curvilinear grid coordinate.
    In principle, there are many different methods to connect the Cartesian 
coordiante and curvilinear coordinate of each grid cell. In APM method, the 
connection is described by an irrotational deformation,
\begin{align}
    x^i=\xi ^\mu \delta ^i _\mu + \Delta x^i,
\end{align}
where
\begin{align}
 \label{eq:disp}
    \Delta x^i=\frac{\partial \phi}{\partial \xi ^ \nu}\delta ^i _\nu .
\end{align}
    This choice of the deformation can minimize mesh distortion and twisting. $\phi$ is called 
the deformation potential, and $\Delta x^i$ the lattice displacement. The deformation potential 
can be given in terms of the continuity equation in curvilinear coordinate,
\begin{align}
 \label{eq:continue_eq}
    \frac{\partial (\sqrt{g} \rho) }{\partial t}+\partial_\mu \left[\rho \sqrt{g} e^\mu _i \left(v^i - \Delta \dot{x}^i \right) \right] =0
\end{align}
where $\sqrt{g} \equiv (\partial x^i / \partial \xi ^ \alpha)$ is the volume element 
and thus $\sqrt{g} \rho)$ represents the particle mass in the volume element under the curvilinear 
coordinate system. $e^i _\mu = \partial \xi ^\mu / \partial x^i$ is the triad. 
$\Delta \dot{x}^i=\delta ^{i\nu}\partial _\nu \dot{\phi}$ is chosen such that the first 
term in equation (\ref{eq:continue_eq}) is zero, resulting in a constant mass per volume 
element. And the velocity field divergence is replaced by the deviation density field 
$\Delta \rho = \bar{\rho}-\rho \sqrt{g}$, which ideally should be zero. Then the 
deformation potential is described in the elliptic equation,
\begin{align}
 \label{eq:li_elip}
    \partial _\mu (\rho \sqrt{g} e^\mu _i \delta^{i\nu}\partial_\nu \Delta \phi)=\Delta \rho
\end{align}
The equation (\ref{eq:li_elip}) can be solved using multigrid algorithm described in Ref. 
\cite{bib:Pen1995,bib:Pen1998} Then the displacement is given by the gradiant of the deformation 
potential as in (\ref{eq:disp}). A 2-D projection of one layer of the deformed grids and the 
original density field on the selected deformed grids are given in Fig. \ref{fig:simandrec}. 
As expeted, there's no grid crossing after reconstruction.
%
\end{section}

