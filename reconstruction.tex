\begin{section}{Reconstruction Technique}
  \label{sec:reconstruction}
  Here we briefly review the MM algorithm used in the new reconstruction technique; for a more
  complete description we refer the reader to \cite{bib:ZhuH2016}.  
  The aim of the MM algorithm is to estimate the displacement field of mass elements in 
  Lagrangian coordinates from their final Eulerian position only, and from this
  estimated displacement field one directly reconstructs the linear density field. The
  general principle is to relate the Eulerian coordinates of a mass element, $x^i$ to
  a curvilinear system, $\xi^\mu$, such that the mass
  per grid cell is approximately constant
  \begin{align}
   \label{eq:const}
    \rho \sqrt{g}=\mathrm{Const.},
  \end{align}
  where $\sqrt{g} \equiv \mathrm{det}\left| e^i_\mu\right|$ is the volume
  element of the coordinate transformation matrix $e^i_\mu = \partial x^i / \partial \xi ^ \mu$. 
  These coordinates are
  related via a {\it deformation} field, which we assume to be a pure
  gradient:
  \begin{align}
    x^i = \xi^\mu \delta^i_\mu + \frac{\partial \phi}{\partial
    \xi^\mu}\delta^{i\mu},
  \end{align}
  and $\phi$ is called the {\it deformation potential} which is chosen to let Eq. \ref{eq:const} hold.  
  Numerically, we iteratively solve for
  the deformation potential via a diffusion equation, 
  \begin{align}
    \label{eq:li_elip}
    \partial _\mu (\rho \sqrt{g} e^\mu _i \delta^{i\nu}
    \partial_\nu \dot{\phi})=\Delta \rho,
  \end{align}
  where $\Delta \rho = \bar{\rho}-\rho \sqrt{g}$ is the difference in density 
  due to displacing the grids. A detailed description 
  of the analytical formulation can be found in the adaptive
  particle mesh and moving mesh hydrodynamics algorithms \cite{bib:Pen1995,bib:Pen1998}.
  Eq.~\ref{eq:li_elip} can be solved by multi-grid
  algorithm\cite{bib:Pen1995,bib:Pen1998,bib:ZhuH2016}.
  Then the displacement field is then given by
  \begin{align}
   \label{eq:disp}
   \bs{\Psi}(\bs{\xi})=\nabla \phi(\bs{\xi}),
  \end{align}
  and the reconstructed density field is given by
  \begin{align}
   \label{eq:delta}
   \delta_R(\bs{\xi})=- \nabla \cdot \bs{\Psi}(\bs{\xi}) = - \nabla^2 \phi(\bs{\xi}). 
  \end{align}

\end{section}

