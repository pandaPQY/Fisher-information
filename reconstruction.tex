\begin{section}{Reconstruction Algorithm}
  \label{sec:reconstruction}
  In this section, we briefly review the MM algorithm; for a more
  complete description we refer the reader to \cite{bib:ZhuH2016}.  
  The aim of the MM algorithm is to estimate the displacement of particles in 
  Lagrangian coordiantes from their final Eulerian position only. The
  general principle is to relate a particle's Eulerian coordinates, $x^i$ to
  a curvilinear system, $\xi^\mu$, in which the number of particles
  per grid cell is approximately constant.  These coordinates are
  related via the so-called deformation, which we assume to be a pure
  gradient:
  \begin{align}
    x^i = \xi^\mu \delta^i_\mu + \frac{\partial \phi}{\partial
    \xi^\mu}\delta^{i\mu}
  \end{align}
  and $\phi$ is called the deformation potential.  Numerially, we 
  can solve iteratively for
  the deformation potential via the diffusion equation, 
  \begin{align}
    \label{eq:li_elip}
    \partial _\mu (\rho \sqrt{g} e^\mu _i \delta^{i\nu}
    \partial_\nu \dot{\phi})=\Delta \rho,
  \end{align}
  where $e^i_\mu = \partial x^i / \partial \xi ^ \mu$ is the
  coordinate transformation matrix,
  $\sqrt{g} \equiv \mathrm{det}\left| e^i_\mu\right|$ is the volume
  element and $\Delta \rho = \bar{\rho}-\rho \sqrt{g}$. Detailed discription 
  of the analytical formulation were presented in \cite{bib:Pen1995,bib:Pen1998}.
  Eq.~\ref{eq:li_elip} can be solved through the use of the multigrid
  algorithm described in \cite{bib:Pen1995,bib:Pen1998,bib:ZhuH2016}.
  The displacement field is then given by
  \begin{align}
   \label{eq:disp}
   \bs{\Psi}(\bs{\xi})=\nabla \phi(\bs{\xi}),
  \end{align}
  and the reconstructed density field is given by
  \begin{align}
   \label{eq:delta}
   \delta_R(\bs{\xi})=- \nabla \cdot \bs{\Psi}(\bs{\xi}) = \nabla^2 \phi(\bs{\xi}). 
  \end{align}

\end{section}

