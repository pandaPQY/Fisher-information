\begin{abstract}
A new reconstruction method using \textit{Moving-mesh} (MM) algorithm from \textit{ Adaptive Particle-mesh} (APM) 
algorithm originally aiming at improving the resolution of N-body simulation is 
recently introduced into comology matter field, which is expected to give a 
better reconstruction from non-linear density fields to linear ones in many 
cases. We are motivated to adapt this method to test the Fisher information 
difference, checking the effect of the reconstruction method. We reconstruct 
136 non-linear density fields given by independent N-body simulations, in 
order to recover some of the information lost in the non-linear regime of 
large-scale structure. Through analyzing the power spectra of both density 
fields from simulations and deformation potentials from reconstructions, we 
find that after reconstruction, the non-linear regime of correlation matrix 
shrinks to $k\sim0.6$. We also find that the cumulative Fisher information 
per $\mathrm{Mpc}^3/h^3$ keeps growing as the linear information until $k\sim 0.3$. 
The saturated value of Fisher cumulative information per $\mathrm{Mpc}^3/h^3$ increase 
from $I \sim 2.5 \times 10^{-5}/(\mathrm{Mpc}^3/h^3)$ to $I \sim 10^{-3}/(\mathrm{Mpc}^3/h^3)$ 
at $k \sim 1$. At least 40 times more information is given through the reconstruction.
\end{abstract}
