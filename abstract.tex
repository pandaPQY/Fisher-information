\begin{abstract}
%A new reconstruction method using \textit{Moving-mesh} (MM) algorithm from \textit{ Adaptive Particle-mesh} (APM) 
%algorithm originally aiming at improving the resolution of N-body simulation is 
%recently introduced into comology matter field, which is expected to give a 
%better reconstruction from nonlinear density fields to linear ones in many 
%cases. We are motivated to adapt this method to test the Fisher information 
%difference, checking the effect of the reconstruction method. We reconstruct 
%136 nonlinear density fields given by independent N-body simulations, in 
%order to recover some of the information lost in the nonlinear regime of 
%large-scale structure. Through analyzing the power spectra of both density 
%fields from simulations and deformation potentials from reconstructions, we 
%find that after reconstruction, the nonlinear regime of correlation matrix 
%shrinks to $k\simeq 0.6$. We also find that the cumulative Fisher information 
%per $\mathrm{Mpc}^3/h^3$ keeps growing as the linear information until $k\simeq 0.3$. 
%The saturated value of Fisher cumulative information per $\mathrm{Mpc}^3/h^3$ increase 
%from $I \simeq 2.5 \times 10^{-5}/(\mathrm{Mpc}^3/h^3)$ to $I \simeq 10^{-3}/(\mathrm{Mpc}^3/h^3)$ 
%at $k \simeq 1$. At least 40 times more information is given through the reconstruction.
Reconstruction techniques are commonly used in cosmology to reduce complicated nonlinear 
behaviour to a more tractable linearized system.  We study the \textit{Moving-Mesh} algorithm which 
is expected to perform better than many alternatives as it is based in Lagrangian space.  
To quantify the algorithm's ability to reconstruct linear modes, we study the Fisher information 
presented in 136 N-body simulations before and after reconstruction.  We find that the linear 
scale is pushed to $k\simeq$ 0.3 $h/\mathrm{Mpc}$ after reconstruction. 
 We furthermore find that the translinear plateau of the cumulative 
Fisher information is increased by a factor of $\sim 40$ after reconstruction, from 
$I \simeq 2.5 \times 10^{-5} /(\mathrm{Mpc}^3/h^3)$ to $I \simeq 10^{-3}/(\mathrm{Mpc}^3/h^3)$ at $k \simeq$ 1 $h/\mathrm{Mpc}$. 
We expect this technique to be beneficial to problems such as baryonic acoustic oscillations and cosmic neutrinos 
that rely on an accurate disentangling of nonlinear evolution from underlying linear effects.
\end{abstract}
