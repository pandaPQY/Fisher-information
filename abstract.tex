\begin{abstract}
  Reconstruction techniques are commonly used in cosmology to reduce
  complicated nonlinear behaviour to a more tractable linearized
  system.  We study the Moving-Mesh algorithm which is able to
  consistently compute the displacement field from non-linear density
  fields.  To quantify the algorithm's ability to reconstruct linear
  modes, we study the Fisher information presented in 130 N-body
  simulations before and after reconstruction.  We find that the
  linear scale is pushed to $k\simeq 0.3$ h/Mpc after
  reconstruction. We furthermore find that the non-linear plateau of
  the cumulative Fisher information is increased by a factor of
  $\sim 50$ after reconstruction, from
  $I \simeq 2.5 \times 10^{-5} /\mathrm{(Mpc/h)}^3$ to
  $I \simeq 1.3 \times 10^{-3}/\mathrm{(Mpc/h)}^3$ at $k \simeq 2.6$ h/Mpc.  This
  result includes the decorrelation between initial and final fields,
  which has been neglected in some previous studies artificially
  improving their performance.  We expect this technique to be
  beneficial to problems such as baryonic acoustic oscillations and
  cosmic neutrinos that rely on an accurate disentangling of nonlinear
  evolution from underlying linear effects.
\end{abstract}
