\begin{abstract}
  Reconstruction techniques are commonly used in cosmology to reduce
  complicated nonlinear behaviours to a more tractable linearized
  system.  We study a new reconstruction technique that uses the
  Moving-Mesh algorithm to estimate the displacement field
  from nonlinear matter distribution. We show the performance of this
  new technique by quantifying its ability to reconstruct linear
  modes. We study the cumulative Fisher information $I(<k_n)$ about  
  the initial matter power spectrum in 130 $N$-body simulations before 
  and after reconstruction, and find that the nonlinear plateau of
  $I(<k_n)$ is increased by a factor of $\sim 50$ after reconstruction, from
  $I \simeq 2.5 \times 10^{-5} /({\rm Mpc}/h)^3$ to
  $I \simeq 1.3 \times 10^{-3}/({\rm Mpc}/h)^3$ at large $k$.
  This result includes the decorrelation between initial and final fields,
  which has been neglected in some previous studies.
We expect this technique to be
  beneficial to problems such as baryonic acoustic oscillations,
  redshift
space distortions and
  cosmic neutrinos that rely on accurately disentangling nonlinear
  evolution from underlying linear effects.
\end{abstract}
