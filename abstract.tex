\begin{abstract}
Reconstruction techniques are commonly used in cosmology to reduce 
complicated nonlinear behaviour to a more tractable linearized system.  
We study the \textit{Moving-Mesh} algorithm which is expected to perform 
better than many alternatives as it is based in Lagrangian space.  
To quantify the algorithm's ability to reconstruct linear modes, we study 
the Fisher information presented in 136 N-body simulations before and after 
reconstruction.  
We find that the linear scale is pushed to $k\simeq$ 0.3 $h/\mathrm{Mpc}$ 
after reconstruction. We furthermore find that the translinear plateau of 
the cumulative Fisher information is increased by a factor of $\sim 40$ 
after reconstruction, from $I \simeq 2.5 \times 10^{-5} /(\mathrm{Mpc}^3/h^3)$ 
to $I \simeq 10^{-3}/(\mathrm{Mpc}^3/h^3)$ at $k \simeq$ 1 $h/\mathrm{Mpc}$. 
This includes the decorrelation between initial and final fields, which has 
been neglected in many previous studies, and we find that the log-normal 
transform in this metric only gains a factor of 4 in information.
We expect this technique to be beneficial to problems such as baryonic 
acoustic oscillations and cosmic neutrinos that rely on an accurate 
disentangling of nonlinear evolution from underlying linear effects.
\end{abstract}
