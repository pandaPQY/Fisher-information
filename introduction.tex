\begin{section}{Introduction}\label{sec:introduction}  

  Two-point statistics provide complete descriptions of Gaussian
  density fields and can be computed efficiently even for large data
  sets.  However, nonlinear gravitational evolution leads to highly
  non-Gaussian matter distributions which require higher order
  statistics to fully characterize.  Such statistics are
  computationally expensive and can be challenging to map to
  cosmological parameters.  To mitigate these difficulties, it is
  common to transform the matter field in a way that hopefully reduces
  non-Gaussianity.  For example, Gaussianization transforms have been
  used to make the logarithmic distribution more Gaussian
  \cite{bib:Weinberg1992,bib:Mark2009} and wavelet nonlinear Wiener
  filters have been used to separate Gaussian and non-Gaussian
  components of the density field
  \cite{bib:Zhang2011,bib:Yu2012,bib:HarnoisD2013}.
  {\it Reconstruction} techniques \citep{bib:Daniel2007} provide a more
  effective way by converting matter distributions back to an
  earlier stage \citep{bib:HarnoisD2013}.

  The quality of these techniques can be quantified by computing the Fisher
  information \citep{bib:Rimes2006} present in the power spectrum before and after
  reconstruction/Gaussianization.
  \citet{bib:Rimes2006} were the first to study the Fisher information in the nonlinear
  matter power spectrum calculated from $N$-body simulations.  They
  found that the cumulative information has a plateau on translinear scales
  ($k \simeq 0.2-0.8$ $h$/Mpc) due to strong coupling between Fourier
  modes.  Qualitatively, this means that the power spectra on these
  scales give little additional information.  
  \citet{bib:Ngan2012} applied the linear reconstruction using Zel'dovich approximation 
  on nonlinear density fields, and found that the cumulative Fisher information increases slightly.
  \citet{bib:HarnoisD2013} computed the cumulative Fisher information
  for various Gaussianization methods and their combinations
  and found that while mode coupling is reduced, there is not
  necessarily an improvement in the cross correlation between the
  initial density fields and the final nonlinear ones. 

  In studies of Baryon Acoustic Oscillations (BAO), density fields are
  subjected to reconstruction which partially inverts nonlinear
  evolution by applying the negative Zel'dovich displacement field \cite{bib:Eisenstein2007}.
  The linear density field is
  typically estimated via Lagrangian perturbation theory (LPT)
  using the linear Zel'dovich displacement \cite{bib:Zel1970}.
  \citet{bib:Yu2016} studied the curl-free, or $E$-mode, component of 
  the exact displacement field in Lagrangian
  space and found that the linear density field can be well recovered by
  the $E$-mode displacement field.
  This $E$-mode reconstruction therefore provides a theoretical 
  target for other reconstruction techniques to be compared with.
  Recently \citet{bib:Zhu2016,bib:ZhuH2016} described a new reconstruction 
  technique using the Moving-Mesh
  algorithm (MM), first described in \cite{bib:Pen1995,bib:Pen1998},
  to effectively estimate $\bs{\Psi}(\bs{q})$ from only nonlinear
  density fields.  They further showed that even though shell-crossing
  and vorticity are not recovered, linear density modes are still 
  recovered up to scales relevent to the BAO.

  In this paper, we compute the Fisher information recovered after
  using this new reconstruction scheme on 130 independent $N$-body
  simulations, and compared with other methods and unreconstructed
  fields.  The paper is organized as follows.  In \S
  \ref{sec:reconstruction}, we briefly
  describe the computation of the displacement potential using MM
  algorithm. In \S \ref{sec:simulation}, we describe  
  the simulations, implementation of the reconstruction and compare 
  the power spectra and cross correlations before and after reconstruction.  
  In \S \ref{sec:fisherinfo}, we further compute the correlation matrix 
  and Fisher information before and
  after reconstruction.  Finally, in \S \ref{sec:conclusion}, we summarize 
  our results and 
  discuss the effectiveness of the reconstruction.
 % and its potential a pplications.


\end{section}

