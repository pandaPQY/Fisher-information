\begin{section}{Introduction}\label{sec:introduction}  

The power spectrum is widely used in modern cosmology to measure the matter
fluctuations. In the early universe, initial Gaussian density fields
can be completely described by the power spectrum, or the two-point
statistics. However, gravitational instability and nonlinear large scale
structure (LSS) formation make the matter distribution highly
non-Gaussian, and the galaxy distribution also follows this non-Gaussian
distribution. In these cases one needs to compute higher statistics
which are computationally more expensive and more difficult to
interpret into initial cosmological parameters. Fisher information
is usually used to quantify the amount of independent information
that is contained in the power spectrum estimation.

Rimes and Hamilton \citep{bib:Rimes2006} first study the Fisher
information contained in the matter power
spectrum given by $N$-body simulations, and find that there is a
plateau on translinear scales $(k \simeq 0.2-0.8\ h/{\rm Mpc})$,
which shows that on these scales, there is a strong coupling of
Fourier modes. Thus the power spectrum on smaller scales, gives little
additional independent information.
%\tcb{Apart from numerical trend, Fisher information was also estimated in data from observation. For example, Lee and Pen \cite{bib:Lee2008} measured the information content in the galaxy angular power spectrum with the help of the Rimes-hamilton technique, and also found that the information saturation. }

There are many approaches to recover the lost information in the
power spectrum of the matter density field, by transforming the final
density field into a more Gaussian, early stage density field.
For example, Gaussianization transforms are commonly used
\cite{bib:Weinberg1992,bib:Mark2009} to make the logarithmic
distribution more Gaussian. Nonlinear Wiener filters are used
in wavelet space to Gaussianize the fields and can also improve
the Fisher information \cite{bib:Zhang2011,bib:Yu2012,bib:HarnoisD2013}.
It is shown in \cite{bib:HarnoisD2013} that, although these methods
or their combinations may have different abilities to recover
the Fisher information, by means of reducing the mode coupling
and variances in the auto power spectrum of Gaussianized
density fields, they do not necessarily improve the cross correlation
 between the initial density
field and the final density field, and thus result in a smearing
out of the baryonic acoustic oscillations (BAO) peak in the two-point
correlation function. If one is concerned about mapping
the initial conditions to final conditions (e.g. measurement of 
BAO) these methods are unable to
extract valid information from initial conditions, at least in
the cross power spectrum between initial and final conditions.

Reconstruction techniques (including the one described in 
\cite{bib:HarnoisD2013}) are able to increase the Fisher information
while also improving the cross correlation with the initial conditions
and sharpening the BAO peak. It is based on the coupling of linear
density field $\delta_L(\bs{q},t_0)$ to the displacement field
$\bs\Psi(\bs{q})$ (first derived by \cite{bib:Zel1970}), which
is estimated by a smoothed final density field.
\cite{bib:Zhu2016} shows a new method in the estimation of displacement
field in 1-dimensional (1D), according to which the 1D linear density
field is reconstructed in Lagrangian space and successfully improves 
the BAO measurement. In 3D cases, it is nontrivial to estimate
the displacement field, but \cite{bib:Yu2016} shows that the displacement
field given by $N$-body simulations can be used to recover $\delta_L$.

In this paper we generalize the displacement field estimation method
from 1D \citep{bib:Zhu2016} to 3D, reconstruct $\delta_L$ and study
the Fisher information recovery in $\delta_L$. Here, the displacement field
estimation is done by a \textit{Moving-mesh} (MM) algorithm, which is based on
the \textit{Adaptive Particle-mesh} (APM) simulation algorithm \citep{bib:Pen1995,bib:Pen1998}.



%\tcr{Neyrinck, Szapudi and Rimes \cite{bib:Mark2006} argued that more information could be extracted on nonlinear scales if the masses of the largest haloes in a survey are known. Neyrinck, Szapudi and Szalay \cite{bib:Mark2009} found that nonlinearities in the dark-matter power spectrum are dramatically smaller if the density field first undergoes a logarithmic mapping, yielding 10 times more cumulative signal-to-noise at $z = 0$. Zhang et al. \cite{bib:Zhang2011} suggested that using Wavelet Wiener filter to seperate Gaussian and non-Gaussian structure in wavelet space is possible to increase the Fisher information by a factor of three before reaching the translinear plateau. Similar steps was also done in the angular power spectrum of weak lensing \cite{bib:Yu2012} and the result showed that there's three times more information compared to that of logarithmic mapping. Later, Neyrinck \cite{bib:Mark2014} applied Gaussianizing transformation method to cosmology and found that it can greatly multiply the Fisher information. Simpson, Heavens and Heymans utilized Clipping \cite{bib:Fergus2013} technique on the matter density field and found it increased the number of useful Fourior modes by more than two orders of magnitude.}


% \tcb{ Pen \cite{bib:Pen1995,bib:Pen1998} introduced new N-bosy algorithm call Adapted Particle-mesh (APM), which scaled linearly with the number of particles for the computational effort per time step, aiming at offering higher resolution. This method can be used to reconstruct the matter density field and is hopeful for tracing a nonlinear density field back to it's linear part\cite{bib:Zhu2016}.}

  This paper is organized as follows. 
  In Section \ref{sec:simulation}, we present the main steps of the N-body simulation code that are used to simulate the dark matter density fields.
  In Section \ref{sec:reconstruction}, we briefly describe the reconstruction algorithm.
  In Section \ref{sec:fisherinfo}, we calculate and compare the power spectra, correlation matrix and Fisher information given by simulation and reconstruction.
  The discussion and conclusion are presented in Section \ref{sec:conclusion}

\end{section}
