\begin{section}{Introduction}\label{sec:introduction}  

  Two-point statistics provide a complete description of Gaussian
  density fields and can be computed efficiently even for large data
  sets.  However, non-linear gravitational evolution leads to highly
  non-Gaussian matter distributions which require higher order
  statistics to fully characterize.  Such statistics are
  computationally expensive and can be challenging to relate to
  cosmological parameters.  To mitigate these difficulties, it is
  common to transform the matter field in a way that hopefully reduces
  non-Gaussianity.  For example, Gaussianization transforms have been
  used to make the logarithmic distribution more Gaussian
  \cite{bib:Weinberg1992,bib:Mark2009} and Wavelet Non-Linear Wiener
  filters have been used to separate Gaussian and non-Gaussian
  components of the density field
  \cite{bib:Zhang2011,bib:Yu2012,bib:HarnoisD2013}.

  The success of techniques can be quantified by computing the Fisher
  information presented in the power spectrum before and after
  reconstruction.  \citet{bib:Rimes2006}
  were the first to study the Fisher information in the non-linear
  matter power spectrum calculated from N-body simulations.  They
  found that the information has a plateau on translinear scales
  ($k \simeq 0.2-0.8\ h/{\rm Mpc}$) due to strong coupling of Fourier
  modes.  Qualitatively, this means that the power spectrum on these
  scales gives little additional information.  
  However, \citet{bib:HarnoisD2013} computed the Fisher information
  for various Gaussianization methods (and combinations of methods)
  and found that while mode coupling is reduced, there is not
  necessarily an improvement in the cross correlation between the
  initial Gaussian density field and the final non-linear one. 

  In studies of Baryon Acoustic Oscillations (BAO), density fields are
  subjected to {\it reconstruction} which partially inverts non-linear
  evolution by applying a negative Zel'Dovich displacement field \cite{bib:Eisenstein2007}.
  The linear density field is
  typically estimated via Lagrangian perturbation theory (LPT)
  using the linear Zel'Dovich displacement $-\nabla_q\cdot\bs\Psi(\bs q)$
  with respect to initial coordiantes $\bs{q}$ \cite{bib:Zel1970}.
  Recently, \citet{bib:Zhu2016} described how to use the Moving-Mesh
  algorithm (MM), first described in \cite{bib:Pen1995,bib:Pen1998},
  to efficiently estimate $\bs{\Psi}(\bs{q})$ from non-linear
  density fields.  They further showed that even though shell-crossing
  and vorticity are not recovered, linear density modes are still 
  recovered up to scales relevent to the BAO.

  In this paper, we compute the Fisher information recovered after
  using this reconstruction scheme on 130 independent N-body
  simulations.  The paper is organized as follows.  In \S
  \ref{sec:reconstruction}, we briefly
  describe the computation of the displacement potential using MM
  algorithm. In \S \ref{sec:simulation}, we describe  
  the simulations, implementation of the reconstruction and compare 
  the power spectra and cross correlations before and after reconstruction.  
  In \S \ref{sec:fisherinfo}, we futher compute the correlation matrix 
  and Fisher information before and
  after reconstruction.  Finally, in \S \ref{sec:conclusion}, we summarize 
  our results and 
  discuss the effectiveness of the reconstruction and its potential
  uses.


\end{section}

