\begin{section}{Introduction}\label{sec:introduction}  

The power spectrum is widely used in modern cosmology to measure the matter
fluctuations. In the early universe, initial Gaussian density fields
can be completely described by the power spectrum, or the two-point
statistics. However, gravitational instability and nonlinear large scale
structure (LSS) formation make the matter distribution highly
non-Gaussian, and the galaxy distribution also follows this non-Gaussian
distribution. In these cases one needs to compute higher statistics
which are computationally more expensive and more difficult to
interpret into initial cosmological parameters. Fisher information
is usually used to quantify the amount of independent information
that is contained in the power spectrum estimation.

Rimes and Hamilton \citep{bib:Rimes2006} first study the Fisher
information contained in the matter power
spectrum given by $N$-body simulations, and find that there is a
plateau on translinear scales $(k \simeq 0.2-0.8\ h/{\rm Mpc})$,
which shows that on these scales, there is a strong coupling of
Fourier modes. Thus the power spectrum on smaller scales, gives little
additional independent information.

There are many approaches to recover the lost information in the
power spectrum of the matter density field, by transforming the final
density field into a more Gaussian, early stage density field.
For example, Gaussianization transforms are commonly used
\cite{bib:Weinberg1992,bib:Mark2009} to make the logarithmic
distribution more Gaussian. Nonlinear Wiener filters are used
in wavelet space to Gaussianize the fields and can also improve
the Fisher information \cite{bib:Zhang2011,bib:Yu2012,bib:HarnoisD2013}.
It is shown in \cite{bib:HarnoisD2013} that, although these methods
or their combinations have different abilities to recover
the Fisher information, by means of reducing the mode coupling
and variances in the auto power spectrum of Gaussianized
density fields, they do not necessarily improve the cross correlation
 between the initial density
field and the final density field, and thus result in a smearing
out of the baryonic acoustic oscillations (BAO) peak in the two-point
correlation function. If one concerns about mapping
the initial conditions to final conditions (e.g. measurement of 
BAO) these methods are unable to
extract valid information from initial conditions, at least in
the power spectrum.

In the linear Lagrangian perturbation theory (LPT), the
negative divergence of the the displacement field
$-\nabla_q\cdot\bs\Psi(\bs q)$ respect to Lagrangian coordinates
$\bs q$ gives the estimated linear density field
\citep{bib:Zel1970} and highly
correlates with $\delta_L$ \citep{2013MNRAS.428..141N,2014PhRvD..89h3515C,bib:Yu2016}. Lagrangian space reconstruction of 
$\delta_L$ can be done if one can estimate the unobservable 
$\bs\Psi(\bs{q})$. The moving-mesh (MM) algorithm, originally 
applied in the adapted particle mesh (APM) $N$-body algorithm
\citep{bib:Pen1995} and moving mesh hydrodynamics (MMH) algorithm
\citep{bib:Pen1998} can be applied to estimate $\bs\Psi(\bs{q})$
for any given density field \citep{bib:ZhuH2016}. This algorithm
simplifies to a reordering of mass elements in 1-dimensional (1D)
case \citep{bib:Zhu2016}. It is shown that, even the nonlinear
shell-crossing and  vorticity part of $\bs\Psi(\bs{q})$ cannot be
recovered, these terms are subdominant and one can still recover
the information on BAO scales \citep{bib:ZhuH2016}.


In this paper, we \tcb{[anything new you are doing on high resolution
simulations, new density assignment methods etc.]} and study
the Fisher information recovery in $\delta_L$. The rest of
this paper is organized as follows. 
In Section \ref{sec:reconstruction}, we briefly describe the reconstruction algorithm.
In Section \ref{sec:simulation}, we present the main steps of the $N$-body 
simulation code that are used to simulate the dark matter density fields, 
and the result of running the reconstruction code.
In Section \ref{sec:fisherinfo}, we calculate and compare the power spectra, 
correlation matrix and Fisher information given by simulation and reconstruction.
The discussion and conclusion are presented in Section \ref{sec:conclusion}

\end{section}
