\begin{section}{Introduction}
  \label{sec:introduction}
  
  Power spectrum of matter density field is essential in modern cosmology research since it is related to many cosmology parameters. Many dedicated experiments attemp to constrain some specific parameters. 

  Fisher information is widely used for making predictions for the errors and covariances of parameter estimates. Rimes and Hamilton\cite{Rimes2006} first studied the Fisher information as a function of scale contained in the matter power spectrum given by N-body simulation, and find that there's a plateau at translinear scales $(k \simeq 0.2-0.8 h\mathrm{Mpc}^{-1})$, which showed that in this region, power spectrum contains little information over and above that in the linear power spectrum. Aprt from numerical trend, Fisher information was also estimated in data from observation. For example, Lee and Pen \cite{bib:Lee2008} measured the information content in the falaxy angular power spectrum with the help of the Rimes-hamilton technique, and also found that the informtation saturation. 
  
  After that, many approaches were put forward to recover parts of the lost information in the power spectrum of matter density field, aiming at the information function closer to that of the power spectrum for linear density fields. Neyrinck, Szapudi and Rimes \cite{bib:Mark2006} argued that more information could be extracted on non-linear scales if the masses of the largest haloes in a survey are known. Neyrinck, Szapudi and Szalay \cite{bib:Mark2009} found that nonlinearities in the dark-matter power spectrum are dramatically smaller if the density field first undergoes a logarithmic mapping, yielding 10 times more cumulative signal-to-noise at $z = 0$. Zhang et al. \cite{bib:Zhang2011} suggested that using Wavelet Wiener filter to seperate Gaussian and non-Gaussian structure in wavelet space is possible to increase the Fisher information by a factor of three before reaching the translinear plateau. Similar steps was also done in the angular power spectrum of weak lensing \cite{bib:Yu2012} and the result showed that there's three times more information compared to that of logarithmic mapping. Later, Neyrinck \cite{bib:Mark2014} applied Gaussianizing transformation method to cosmology and found that it can greatly multiply the Fisher information. 

  Pen \cite{bib:Pen1995}\cite{bib:Pen1998} introduced new N-bosy algorithm call Adapted Particle-mesh (APM), which scaled linearly with the number of particles for the computational effort per time step, aiming at offering higher resolution. This method can be used to reconstruct the matter density field and is hopeful for tracing a non-linear density field back to it's linear part\cite{bib:Zhu2016}.

  This paper is organized as follows. 
  In Section \ref{sec:simulation}, we present the main steps of the N-body simulation code that was used to simulate the dark matter density fields.
  In Section \ref{sec:reconstruction}, we briefly describe the reconstruction algorithm.
  In Section \ref{sec:fisherinfo}, we calculate and compare the power spectra, correlation matrix and Fisher information given by simulation and reconstruction.
  Conclusion and discussion are presented in Section \ref{sec:conclusion}

\end{section}
