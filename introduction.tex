\begin{section}{Introduction}\label{sec:introduction}  

  Two-point statistics provide a complete description of Gaussian
  density fields and can be computed efficiently even for large data
  sets.  However, non-linear gravitational evolution leads to highly
  non-Gaussian matter distributions which require higher order
  statistics to fully characterize.  Such statistics are
  computationally expensive and can be challenging to relate to
  cosmological parameters.  To mitigate these difficulties, it is
  common to transform the matter field in a way that hopefully reduces
  non-Gaussianity.  For example, Gaussianization transforms have been
  used to make the logarithmic distribution more Gaussian
  \cite{bib:Weinberg1992,bib:Mark2009} and Wavelet Non-Linear Wiener
  filters have been used to separate Gaussian and non-Gaussian
  components of the density field
  \cite{bib:Zhang2011,bib:Yu2012,bib:HarnoisD2013}.

  The success of techniques can be quantified by computing the Fisher
  information present in the power spectrum before and after
  reconstruction.  For linear fields, the Fisher information is simply
  proportional to the number of modes ($k^3$).  \citet{bib:Rimes2006}
  were the first to study the Fisher information in the non-linear
  matter power spectrum calculated from N-body simulations.  They
  found that the information has a plateau on translinear scales
  ($k \simeq 0.2-0.8\ h/{\rm Mpc}$) due to strong coupling of Fourier
  modes.  Qualitatively, this means that the power spectrum on small
  scales gives little additional information.  
  However, \citet{bib:HarnoisD2013} computed the Fisher information
  for various Gaussianization methods (and combinations of methods)
  and found that while mode coupling is reduced, there is not
  necessarily an improvement in the cross correlation between the
  initial Gaussian density field and the final non-linear one. 

  In studies of Baryon Acoustic Oscillations (BAO), density fields are
  subjected to {\it reconstruction} which partially inverts non-linear
  evolution by applying a negative displacement field.  This field is
  typically computed via Lagrangian perturbation theory (LPT)
  using the linear Zel'Dovich displacement, $-\nabla_q\cdot\bs\Psi(\bs q)$
  with respect to initial coordiantes $\bs{q}$ \citep{bib:Zel1970}.
  Recently, \citet{bib:Zhu2016} described how to use the Moving Mesh
  algorithm (MM), first described in \citep{bib:Pen1995,bib:Pen1998},
  to consistently compute $\bs\Psi(\bs{q})$ even for non-linear
  density fields.  They further showed that even though shell-crossing
  and vorticity are not recovered, information is still recovered on
  scales relevent tot he BAO.

  In this paper, we compute the Fisher information recovered after
  using this reconstruction scheme on 136 independent N-body
  simulations.  The paper is organized as follows.  In \S
  \ref{sec:reconstruction} and \ref{sec:simulation}, we briefly
  describe the computation of the displacement potential using MM
  reconstruction and the N-body simulations used for the Fisher
  information computation.  In \S \ref{sec:fisherinfo}, we compute the
  power spectra, correlation matrix and Fisher information before and
  after reconstruction.  Finally, in \S \ref{sec:conclusion}, we
  discuss the effectiveness of the reconstruction and its potential
  uses.


\end{section}

  % The power spectrum is widely used in modern cosmology to measure the
  % matter fluctuations. In the early universe, initial Gaussian density
  % fields can be completely described by the power spectrum, or the
  % two-point statistics. However, gravitational instability and
  % nonlinear large scale structure (LSS) formation make the matter
  % distribution highly non-Gaussian, and the galaxy distribution also
  % follows this non-Gaussian distribution. In these cases one needs to
  % compute higher statistics which are computationally more expensive
  % and more difficult to interpret into initial cosmological
  % parameters. Fisher information is usually used to quantify the
  % amount of independent information that is contained in the power
  % spectrum estimation.

  % Rimes and Hamilton \citep{bib:Rimes2006} first study the Fisher
  % information contained in the matter power spectrum given by $N$-body
  % simulations, and find that there is a plateau on translinear scales
  % $(k \simeq 0.2-0.8\ h/{\rm Mpc})$, which shows that on these scales,
  % there is a strong coupling of Fourier modes. Thus the power spectrum
  % on smaller scales, gives little additional independent information.

  % There are many approaches to recover the lost information in the
  % power spectrum of the matter density field, by transforming the
  % final density field into a more Gaussian, early stage density field.
  % For example, Gaussianization transforms are commonly used
  % \cite{bib:Weinberg1992,bib:Mark2009} to make the logarithmic
  % distribution more Gaussian. Nonlinear Wiener filters are used in
  % wavelet space to Gaussianize the fields and can also improve the
  % Fisher information \cite{bib:Zhang2011,bib:Yu2012,bib:HarnoisD2013}.
  % It is shown in \cite{bib:HarnoisD2013} that, although these methods
  % or their combinations have different abilities to recover the Fisher
  % information, by means of reducing the mode coupling and variances in
  % the auto power spectrum of Gaussianized density fields, they do not
  % necessarily improve the cross correlation between the initial
  % density field and the final density field, and thus result in a
  % smearing out of the baryonic acoustic oscillations (BAO) peak in the
  % two-point correlation function. If one concerns about mapping the
  % initial conditions to final conditions (e.g. measurement of BAO)
  % these methods are unable to extract valid information from initial
  % conditions, at least in the power spectrum.

  % In the linear Lagrangian perturbation theory (LPT), the negative
  % divergence of the the displacement field
  % $-\nabla_q\cdot\bs\Psi(\bs q)$ respect to Lagrangian coordinates
  % $\bs q$ gives the estimated linear density field \citep{bib:Zel1970}
  % and highly correlates with $\delta_L$
  % \citep{2013MNRAS.428..141N,2014PhRvD..89h3515C,bib:Yu2016}. Lagrangian
  % space reconstruction of $\delta_L$ can be done if one can estimate
  % the unobservable $\bs\Psi(\bs{q})$. The moving-mesh (MM) algorithm,
  % originally applied in the adapted particle mesh (APM) $N$-body
  % algorithm \citep{bib:Pen1995} and moving mesh hydrodynamics (MMH)
  % algorithm \citep{bib:Pen1998} can be applied to estimate
  % $\bs\Psi(\bs{q})$ for any given density field
  % \citep{bib:ZhuH2016}. This algorithm simplifies to a reordering of
  % mass elements in 1-dimensional (1D) case \citep{bib:Zhu2016}. It is
  % shown that, even the nonlinear shell-crossing and vorticity part of
  % $\bs\Psi(\bs{q})$ cannot be recovered, these terms are subdominant
  % and one can still recover the information on BAO scales
  % \citep{bib:ZhuH2016}.

  % In this paper, we \tcb{[anything new you are doing on high
  %   resolution simulations, new density assignment methods etc.]} and
  % study the Fisher information recovery in $\delta_L$. The rest of
  % this paper is organized as follows.  In Section
  % \ref{sec:reconstruction}, we briefly describe the reconstruction
  % algorithm.  In Section \ref{sec:simulation}, we present the main
  % steps of the $N$-body simulation code that are used to simulate the
  % dark matter density fields, and the result of running the
  % reconstruction code.  In Section \ref{sec:fisherinfo}, we calculate
  % and compare the power spectra, correlation matrix and Fisher
  % information given by simulation and reconstruction.  The discussion
  % and conclusion are presented in Section \ref{sec:conclusion}