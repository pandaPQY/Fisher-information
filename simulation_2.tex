\begin{section}{Inplementation}
  \label{sec:simulation}
We use the \textsc{CUBEP$^3$M} \cite{bib:Harnois2013} to 
run 136 simulations with a box size of 300 $h^{-1}$Mpc, 
afine resolution of $1024^3$ cells and $512^3$ totally particles. 
The initial conditions are computed using the transfer function 
given by CAMB \cite{bib:Lewis2000} and then propogating the power 
linearly back to $z=100$ with a growth factor. 
The Zel'dovich approximation is used to calculate the displacement 
and velocity fields, which are assigned to the particles. 
The cosmological parameters used are $\Omega_M=0.32$, 
$\Omega_{\Lambda}=0.679$, $h=0.67$, $\sigma_8=0.83$, and $n_s=0.96$. 
Different seeds are used to produce the initial conditions for 
different simulations so that they are 
independent to each other.

Then we run the MM reconstruction code on the nonlinear density fields
from simulation in a resolution of ng $=128$ per dimension. The multigrid 
algorithmis iterated for 1000 times in the result of the root mean 
square decreasing from $\sim$ 4.5 to $\sim$ 0.2.
A 2D projection of one layer of the deformed grids and the
original density field on the grid are given in Fig.\ref{fig:simandrec}.
As expected, there is no grid crossing after reconstruction.


\begin{figure*}[t!]
\centering
 \includegraphics[width=0.9\textwidth]{fig1.pdf}
   \caption{
The 2-D projection of the deformed grid of a sample $N$-body simulations 
is shown as curved white lines. The $\delta+1$ field on the deformed 
grid is shown underneath.}
 \label{fig:simandrec}
\end{figure*}

\end{section}

