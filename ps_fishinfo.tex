\begin{section}{Fisher Information Content}
  \label{sec:fisherinfo}
  % The cross power spectrum, $P_{ij}(k)$, is defined as
  % \begin{align}
  %   \langle \delta_i \left( \bm{k} \right) \delta_j \left( \bm{k'}\right) \rangle =
  %   \left( 2\pi \right) ^3 P_{ij} \left( k \right) \delta_D \left( \bm{k}-\bm{k'} \right),
  % \end{align}
  % where $\delta_{i}$ and $\delta_j$ are density contrasts and
  % $\delta_D$ is the Dirac delta funciton. We typically consider
  % instead the dimensionless power spectrum, $\Delta_{ij}^2(k)$, defined as
  % \begin{align}
  %   \Delta_{ij}^2(k) \equiv \frac{k^3 P_{ij} \left( k \right)}{2\pi ^2}.
  % \end{align}
  % The power spectra of the mass distributions are calculated using the
  % \enquote{\textit{Nearest Grid Point}} (NGP) mass assignment scheme.
  % In Fig.\ref{fig:cp}(a) we plot the mean cross correlation function,
  % $r=P_{\delta \delta_L}/\sqrt{P_\delta P_{\delta_L}}$ of the
  % nonlinear and the linear power spectrum, and the reconstructed and
  % linear power spectrum respectively.  The wave number where the cross
  % correlation drops to a half increases from $k\simeq$ 0.2
  % $h/\mathrm{Mpc}$ to $k \simeq$ 0.6 $h/\mathrm{Mpc}$ after the
  % reconstruction.  To qualify the improvement of cross correlation
  % better, we compute the damping factors $\mathcal{D}(k)=r^4$ fitting
  % the Gaussian BAO damping models
  % $\mathcal{D}(k)=\mathrm{exp}(-k^2 \Sigma^2/2)$.  In
  % Fig.\ref{fig:cp}(a) we plot $\mathcal{D}_\delta^{1/4}$ ($\Sigma =$
  % 11.3 $\mathrm{Mpc}/h$) and $\mathcal{D}_{\delta_R}^{1/4}$
  % ($\Sigma = $ 3.9 $\mathrm{Mpc}/h$) over $r_{\delta\delta_L}$ and
  % $r_{\delta_R\delta_L}$.  We also plot $\mathcal{D}(k)^{1/4}$ that
  % match cross correlation function after $E$-mode displacement
  % reconstruction in \cite{bib:Yu2016} ($\mathrm{ng}=512$, box size $=$
  % 400 $\mathrm{Mpc}/h$, $\Sigma =$ 1.3 $\mathrm{Mpc}/h$), and MM
  % reconstruction in a higer resolution in \cite{bib:ZhuH2016}
  % ($\mathrm{ng}=512$, box size $=$ 600 $\mathrm{Mpc}/h$, $\Sigma =$
  % 2.6 $\mathrm{Mpc}/h$).  We find that in higher resolution, the
  % reconstruction gives a cross correlation damping at smaller scale.
  % And it's expected that the $E$-mode displacement reconstruction
  % gives a reconstructed power spectrum more correlated with the
  % initial one, since it completely picks out the irrotational
  % component of the real displacement field in $N$-body simulation,
  % while through MM reconstruction, the difference between the
  % reconstructed displacement and the real displacement still contains
  % an irrotational component, which is also correlated with the linear
  % power spectrum.  In Fig.\ref{fig:cp}(b) we plot the linear power
  % spectrum, and mean power spectrum (with error bars) of 136 nonlinear
  % density fields and reconstruced density fields simply given by
  % $\delta_R=-\nabla^2\phi$.  The reconstructed power spectrum drops at
  % nonlinear scale ($k \gtrsim$ 0.3 $h/\mathrm{Mpc}$) since the
  % reconstructed density fields are totally irrotational.  The result
  % is similar to that of $E$-mode displacement reconstruction described
  % in \cite{bib:Yu2016}, in which the reconstructed power spectrum
  % drops, but in a different scale and at a different speed.
  % \begin{figure*}
  %   \centering
  %   \includegraphics[width=0.455\textwidth]{fig2a.pdf}
  %   \includegraphics[width=0.48\textwidth]{fig2b.pdf}
  %   \caption{(a) The cross correlation function (solid lines)
  %     $r_{\delta\delta_L}$ (blue) and $r_{\delta_R\delta_L}$ (red),
  %     and BAO damping models (dash lines).  (b) The dimensionless
  %     power spectrum computed via linear theory (black), the mean
  %     value of 136 $N$-body simulations with $1\sigma$ error bars
  %     (blue), and reconstruction of the simulations (red).}
  %   \label{fig:cp}
  % \end{figure*}

  Mathamatically, the Fisher information $I$ of the initial scale
  invariant matter power spectrum, $A$, is defined as
  \begin{align}
    I_A \equiv -\left\langle \frac{\partial ^2 \mathrm{ln \Largr}}{\partial A ^2}\right\rangle,
    \label{eq:fisherdefine}
  \end{align}
  in which $\Largr$ denotes the likelihood \cite{bib:Tegmark1997}.  For Gaussian
  fluctuations, the likelihood depends on parameters only through the
  power spectrum $P(k)$, so $I$ can be written as 
  \begin{align}
    I_A = - \left\langle \sum_{k,k'} \frac{\partial \mathrm{ln} P(k)}{\partial \mathrm{ln} A} 
    \frac{\partial ^2 \mathrm{ln \Largr}}{\partial \mathrm{ln} P(k) \partial \mathrm{ln} P(k')}
    \frac{\partial \mathrm{ln} P(k')}{\partial \mathrm{ln} A}\right\rangle,
    \label{eq:fisherdef2}
  \end{align}
  in which the angle bracket denotes the average of many realizations
  of the power spectrum \cite{bib:Rimes2006}.

  Eq. \ref{eq:fisherdef2} can be written in a simpler
  form in two aspects.   First, we can simplify the derivative terms
  $\partial P/\partial A$.  For any density field $\delta_i$, we can
  conveniently decompose it into linear and non-linear components
  \begin{align}
    \delta_(k) = b (k) \delta _L (k) + \delta_{N}(k),
    \label{eq:decompose}
  \end{align}
  in which $\delta_L$ denotes the linear density field, $b(k)$ is the
  bias and $\delta_{N}(k)$ is defined such that the correlation
  $\langle \delta_L(k)\delta_{N}(k) \rangle$ is zero.  If we correlate
  $\delta_i$ and $\delta_L$,
  \begin{align}
    \langle \delta_i(k)\delta_L(k) \rangle = b(k) \langle \delta_L(k)\delta_L(k) \rangle,
    \label{eq:correlating}
  \end{align} 
  we can solve for $b$ as
  \begin{align}
    b (k) = \frac{P _{iL}(k)}{P_{LL}(k)}.
    \label{eq:bofk}
  \end{align}
  To find the non-linear term, we correlate $\delta_i$ with itself,
  \begin{align}
    \langle \delta_i(k) \delta_i(k) \rangle = 
    b^2(k) \langle \delta_L(k) \delta_L(k) \rangle + \langle \delta_{N}(k)\delta_{N}(k) \rangle,
  \end{align}
  and find
  \begin{align}
    P_{ii}^2 (k) = b^2(k) P_{LL} (k) + P_{NN} (k).
    \label{eq:powerdecompose}
  \end{align}
  With the help of Eq. \ref{eq:bofk} and Eq. \ref{eq:powerdecompose},
  we can replace the partial derivatives
  $\partial \mathrm{ln} P(k) / \partial \mathrm{ln} A$ in
  Eq. \ref{eq:fisherdef2} with $r^2_{iL}$.
  % \begin{align}
  %   \frac{A}{P(k)}\frac{\partial P(k)}{\partial A}=
  %   \frac{P_{\delta \delta_L}^2(k)}{P_\delta(k) P_{\delta_L}(k)},
  % \end{align}
  % which is just the square of the cross correlation function
  % $r ^2 (k)$, of $\delta$ and $\delta_L$.

  % The second partial derivative terms in Eq. \ref{eq:fisherdef2}, the
  % Hessian of the vector $\mathrm{ln} P(k)$, has the expectation value
  % of the Fisher matrix with respect to the log powers. For linear
  % density fields, the Fisher matrix is approximately equal to the
  % inverse of the covariance matrix of power spectrum estimates, which
  % should be diagonal, with diagonal elements equal to number of modes
  % in each wavenumber bin (when considering $\bm{k}$ and $-\bm{k}$ as
  % the same mode).  Thus we can write down a simpler matrix product
  % form of cumulative Fisher information,
  The second step we can make is to simplify
  $\partial ^2 \mathrm{ln \Largr}/\partial \mathrm{ln} P(k) \partial
  \mathrm{ln} P(k')$
  by utilizing the fact that its expectation value is the Fisher
  matrix.  For linear fields, this is equal to the inverse of the
  covariance matrix which is diagonal with elements given by the
  number of modes in each bin.  We can extend this definition to
  non-linear fields, provided we take into account that the covariance
  matrix is no longer diagonal and invert it appropriately.  Thus, we
  can write the Fisher information in terms of matrix multiplication:
  \begin{align}
    I_A \left( < k_n\right) = r^2(k)^{\mathrm{T}} \left[ \mathrm{C^{-1}_{norm}} 
    ( k,k' )\right]_{<k_n} r^2(k') ,
    \label{eq:fisherformulaused}
  \end{align}
  where $\mathrm{C_{norm}}$ is the normalized covariance matrix
  defined as
  \begin{align}
    \mathrm{C_{norm}} \left( k,k' \right)=\frac{\mathrm{Cov}(k,k')}
    {\langle P(k)\rangle\langle P(k')\rangle},
  \end{align}
  $r$ is the mean cross correlation of a given density field with
  linear one and the subscript $<k_n$ indicates the matrix is set to
  zero for modes $k,k'>k_n$.  The covariance matrix is defined as
  \begin{align}
    \mathrm{Cov}\left(k,k'\right)\equiv \frac{\sum_{i,j=1}^{N}\left[ P_i \left( k \right) - 
    \langle P \left( k \right) \rangle \right]\left[ P_j \left( k' \right) - 
    \langle P \left( k' \right)\rangle \right]}{N-1},
  \end{align}
  where $N$ is the total number of simulations and angle brackets are
  values averaged over all simulations.  

  The correlation matrix is the normalized version of the covariance
  matrix:
  \begin{align}
    \mathrm{Corr}\left(k,k'\right)=\frac{\mathrm{Cov}\left(k,k'\right)}
    {\sqrt{\mathrm{Cov}\left(k,k\right)\mathrm{Cov}\left(k',k'\right)}},
  \end{align}
  and represents the correlation between different $k$ modes.  The
  corelation matrices for non-linear and reconstructed power spectra
  are shown in the upper and lower sections of Fig. \ref{fig:corrall}.
  By definition, the correlation matrix is symmetrix with unit
  diagonal allowing us to overlay the two matrices.  For the
  non-linear case, the correlation matrix is almost diagonal linear
  regime $k \lesssim$ 0.07 $h/\mathrm{Mpc}$.  The off-diagonal
  elements are produced by strong mode coupling on non-linear scales
  and the super-survey tidal effect which is small on linear scales
  but dominates in the weakly non-linear regime
  \cite{bib:Kazuyuki2016}.  The correlation matrix for the nonlinear
  power spectra has few negative elements
  ($\mathrm{Corr} \gtrsim -0.1$), which should vanish with more
  simulations \cite{bib:Takahashi2009}.  For the reconstructed
  correlation matrix, the linear regime expands up to $k \simeq 0.3$
  h/Mpc.  However, the number and magnitude of negative off-diagonal
  elements also increases ($\mathrm{Corr} \gtrsim -0.8$).

  \begin{figure}
    \centering
    \includegraphics[width=0.48\textwidth]{fig3.pdf}
    \caption{The correlation matrix as found from 136 non-linear power
      spectra (the upper-left elements) and the reconstructed power
      spectra (the lower-right off-diagonal elements).}
    \label{fig:corrall}
  \end{figure}

  We plot the cumulative Fisher information per volume of the
  nonlinear, linear and reconstructed power spectra in
  Fig. \ref{fig:fisherinfo}(a). The Fisher information of the
  non-linear power spectra drops from the linear one at
  $k \simeq 0.05$ h/Mpc, and has a flat plateau in the translinear
  regime, $k\simeq 0.3$ h/Mpc, with a saturated value of
  $I \simeq 2.5 \times 10^{-5}/(\mathrm{(Mpc/h)}^3)$.  It indicates
  that there is nearly no independent information in the translinear
  regime.  But the information curve of the reconstructed power
  spectra keeps increasing roughly the same as the linear information
  until $k\simeq 0.3$ h/Mpc, and reaches its plateau at $k\simeq 0.8$
  h/Mpc with a value of $I \simeq 10^{-3}/(\mathrm{(Mpc/h)}^3)$, up by
  a factor of 40.  It indicates that the MM reconstructed method can
  strongly recover the lost information within these scales.  We
  compare the Fisher information given by the MM reconstruction method
  with the logarithmic density mapping method \cite{bib:Mark2009} as
  an example to illustrate its strength.  We find that the MM
  reconstruction gives 10 times more information than the logarithmic
  mapping.  In some papers, the cross correlation $r^2$ terms are set
  to unity in Eq. \ref{eq:fisherformulaused}, which artificially
  increases the nonlinear information.  For comparison, we plot this
  case in Fig.\ref{fig:fisherinfo}(b).  We see that the logarithmic
  density mapping information is much higher, but does not correlate
  with the initial conditions.

  % We find that, in
  % this case, the nonlinear information drops from the linear one
  % beginning at the same scale, $k\simeq$ 0.3 $h/\mathrm{Mpc}$, but
  % reaches the saturated value,
  % $I \simeq 4 \times 10^{-5}/(\mathrm{Mpc}^3/h^3)$, at translinear
  % scale, $k \simeq $ 0.2 $-$ 0.8 $h/\mathrm{Mpc}$.  However, the MM
  % reconstructed and logarithmic mapping information is higher than
  % the
  % linear one in the scale $k \simeq$ 0.2 $-$ 0.5 $h/\mathrm{Mpc}$,
  % which is not expected.
  \begin{figure*}
    \includegraphics[width=0.5\textwidth]{fig4a.pdf}
    \includegraphics[width=0.46\textwidth]{fig4b.pdf}
    \centering
    \caption{(a) The cumulative Fisher information per unit volume as
      a function of wavenumber.  The blue line corresponds to the
      non-linear density fields, the red line with squares corresponds
      to the the reconstructed density fields, the dark line
      corresponds to the linear density fields, the cyan line
      corresponds to the logarithmic density mapping, and the circles
      are $\propto k^3$.  Dotted lines correspond 4 and 40 times the
      saturated value of the non-linear Fisher information.  (b) Same
      as (a) except with $r\equiv 1$. The black, blue and cyan lines
      match the results in \cite{bib:Rimes2006,bib:Mark2009}.}
  \label{fig:fisherinfo}
\end{figure*}
\end{section}
