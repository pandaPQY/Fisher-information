\begin{section}{Conclusion and Discussion}
 \label{sec:conclusion}
%    We use the code "CubeP3M" to generate 136 independent dark matter density fields, 
%then give the reconstructed deformation potentials which are pure divergent using MM 
%algorithm. We analyze the power spectra of both the non-linear density fields and 
%reconstructed density fields, after which we give the cross correlation matrix. We find that the power 
%spectra are highly correlated on small scales, since these scales are in non-linear 
%regime. But after reconstruction, the strongly correlated regime shinks from $k\simeq 0.2$ 
%to $k\simeq 0.6$. We also calculate the cumulative information, and find that the plateau 
%of the reconstructed information curve keeps matching the Gausisan information curve until 
%$k \simeq 0.3$ and reach a plateau at $k \simeq 0.8$, rising by a factor of 40 compared to the 
%non-linear information curve.

%   We believe that the reconstructed Fisher information will still increasing to a greater magnitude in smaller scale 
%since the cross correlation of the reconstructed power spectrum with linear increases 
%in a higher resolution analysis \cite{bib:ZhuH2016}.

   The new reconstruction method successfully recovers the lost linear information on 
the mildly non-linear scale, increasing the saturated information from 
$I \simeq 2.5 \times 10^{-5}/(\mathrm{Mpc}^3/h^3)$ to at least$I \simeq 10^{-3}/(\mathrm{Mpc}^3/h^3)$, 
and ushes the non-linear scale to higher $k$. 
The result is better than previous methods 
(e.g. \cite{bib:Mark2006,bib:Mark2009,bib:Zhang2011,bib:Yu2012,bib:Mark2014}), and
we believe that the reconstructed Fisher information will still increasing to a greater magnitude in smaller scale
since the cross correlation of the reconstructed power spectrum with linear increases                
in a higher resolution analysis \cite{bib:ZhuH2016}.
The result in dark matter density fields 
gives a strong motivation to adapt the MM reconstruction to halo fields, neutrino fields, etc, so that 
we have access to the physics in smaller scales. Reconstruction technique are concerned  
to improve cosmology measurements of BAO scale (e.g. \cite{bib:Daniel2007,bib:Martin2015}). 
The MM reconstruction gives the reconstructed displacement on the Lagrangian coordinates transfromed from the 
Eulerian coordiantes. It's successful application on BAO reconstruction in 1-D 
\cite{bib:Zhu2016} and 3-D \cite{bib:ZhuH2016} cosmology provides an intuitive view of the algorithm 
to push forward BAO research.

   The MM reconstruction effectively decomposes the irrotational part and the curl part of the displacement 
field of particles. However, the reconstructed displacement might be greatly different from real displacement in N-body 
simulation, since it is sensitive to the late stage shell-crossing and non-linear process so that the original 
position of some spectific particles are replaced by each other. It's meaningful to compare the irrotational 
displacement field through the MM reocnstruction and that from $E$-mode displacement reconstruction \cite{bib:Yu2016} 
which decomposes completely the irrotational and curl components of the real displacement in N-body simulation. 
Since the MM reconstruction only needs the density field input and gives a large recovering of lost information, 
it's expected to have a good effect on reconstructing the matter density field from observation.
\end{section}
