\begin{section}{Conclusion}
  \label{sec:conclusion}
  The MM reconstruction method successfully recovers the lost linear
  information on mildly nonlinear scales and increases the saturated
  information from $I \simeq 2.5 \times 10^{-5}/({\rm Mpc}/h)^3$
  to at least $I \simeq 1.3 \times 10^{-3}/({\rm Mpc}/h)^3$.  The result is
  better than previous methods,
  e.g. \cite{bib:Mark2006,bib:Mark2009,bib:Zhang2011,bib:Yu2012}.
  %, and
  %may improve further as the correlation coefficient between the
  %reconstructed and linear fields will increase with higher resolution
  %simulations \cite{bib:ZhuH2016}.  This successful result on cold
  %dark matter density fields provides strong motivation to adapt the
  %MM reconstruction scheme to other cosmological fields such as biased
  %tracers like halos and other matter components like baryons and
  %neutrinos.

  \tcb{(i) 50 or 150 times more information is
  quoted as the $k\rightarrow\infty$ case, which is in anyways unachievable.
  Within our scales of interest (BAO)
  $I_{\delta_R}(<k_{\rm BAO}) \simeq I_{\delta_E}(<k_{\rm BAO})$. Thus,
  further optimizing the code does not improve the BAO accuracy. Furthermore,
  the bottleneck is the number density Poisson noise and bias from surveys.
  I hope Yu Yu give more comments here.\\
  (ii) Even assume we optimized (i) and regardless of BAO, 150/5=3 times
  more $I(<k)$ means lowering the error bar by factor of 1.7; whereas 50/1=50 times
  more information means lowering the error bar by 7. The latter is of much more importance.\\
  (iii) if you compare (ii) between MM and standard reconstruction (e.g. reconstruction
  part in \cite{bib:HarnoisD2013}), then you can quantify how much better MM
  is better than standard reconstruction in terms of Fisher information.\\
  (iv) The next step/work by Xin et al. will give a direct answer to (iii)
  in terms of the accuracy of BAO dilation scale.
  }

\end{section}
