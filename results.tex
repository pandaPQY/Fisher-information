\begin{section}{Conclusion}
  \label{sec:conclusion}
  We study the Moving-Mesh algorithm's ability to estimate the underlying displacement 
  field and reconstruct the linear density fields using 130 cosmological $N$-body 
  simulations.  We measure the power spectra and the associated covariance of the 
  nonlinear density fields, and the reconstructed density fields.  We quantify the 
  result by (i) cross-correlating them with the linear density fields, (ii) studying 
  the $k$-mode coupling in the correlation matrix, and (iii) computing the cumulative 
  Fisher information contained in these power spectra.  We also
  compared with the $E$-mode 
  reconstruction and logarithmic density mapping Gaussianization
  techniques.
We find that Moving-Mesh method gives better results than previous works
  (e.g. \cite{bib:Mark2009,bib:Zhang2011,bib:HarnoisD2013}), and on scales 
  relevant to the BAO, our result approaches the optimal $E$-mode reconstruction.  
  Future steps include quantifying Halo Poisson noise and bias contamination 
  from realistic measurements, and determining the quantitative impact on
  BAO and RSD measurements.  

%  From 130 large scale structure $N$-body simulations, 
%  we study the moving-mesh algorithm on the estimation of the underlying displacement
%  field and the reconstruction of linear density fields in Lagrangian
%  space. We measure the power spectra and their covariance of
%  the original nonlinear density fields $\delta_N$, reconstructed density fields $\delta_R$.
%  We quantify the result by (i) cross-correlating them respect to linear density fields $\delta_L$ (Fig. \ref{fig:cp}),
%  (ii) studying the $k$-mode coupling in the covariance matrix (Fig. \ref{fig:corrall}), and (iii)
%  computing the cumulative Fisher information contained
%  in these power spectra (Fig. \ref{fig:fisherinfo}). We also apply $E$-mode reconstruction
%  and logarithmic transformation Gaussianization and compare
%  with the moving-mesh reconstruction result. We show
%  that the moving-mesh method gives better results than previous works
%  (e.g. \cite{bib:Mark2006,bib:Mark2009,bib:Zhang2011,bib:HarnoisD2013}),
%  and on our scales of interest (BAO), the result is comparable to
%  the optimal $E$-mode reconstruction case \citep{bib:Yu2016}.
%  Next steps are to overcome the Poisson and bias contaminations
%  from realities, and to quantify the error propagation onto
%  the BAO measurements.
%
\end{section}
