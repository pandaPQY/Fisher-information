\begin{section}{Conclusion}
  \label{sec:conclusion}
  The MM reconstruction method successfully recovers the lost linear
  information on mildly nonlinear scales and increases the saturated
  information from $I \simeq 2.5 \times 10^{-5}/({\rm Mpc}/h)^3$
  to at least $I \simeq 1.3 \times 10^{-3}/({\rm Mpc}/h)^3$.  The result is
  better than previous methods,
  e.g. \cite{bib:Mark2006,bib:Mark2009,bib:Zhang2011,bib:Yu2012}.
  %, and
  %may improve further as the correlation coefficient between the
  %reconstructed and linear fields will increase with higher resolution
  %simulations \cite{bib:ZhuH2016}.  This successful result on cold
  %dark matter density fields provides strong motivation to adapt the
  %MM reconstruction scheme to other cosmological fields such as biased
  %tracers like halos and other matter components like baryons and
  %neutrinos.

  \tcr{I will rewrite the conclusion.}

  % The result in dark matter density fields gives a strong motivation
  % to adapt the MM reconstruction to halo fields, neutrino fields,
  % etc, so that we have access to the physics in smaller scales.

  % Reconstruction technique are concerned to improve cosmology
  % measurements of BAO scale
  % (e.g. \cite{bib:Daniel2007,bib:Martin2015}).  The successful
  % application of the MM reconstruction on BAO reconstruction in 1D
  % \cite{bib:Zhu2016} and 3D \cite{bib:ZhuH2016} cosmology provide an
  % intuitive view of the algorithm to push forward BAO research.

  % The MM reconstruction effectively decomposes the irrotational part
  % and the curl part of the displacement field of particles. However,
  % the reconstructed displacement might be greatly different from the
  % real displacement in N-body simulation, since it is sensitive to
  % the
  % late stage shell-crossing and nonlinear process so that the
  % original
  % position of some spectific particles are replaced by each other.
  % It
  % is meaningful to compare the irrotational displacement field
  % through
  % the MM reocnstruction and through $E$-mode displacement
  % reconstruction \cite{bib:Yu2016}.  Since the MM reconstruction
  % only
  % needs the density field input and gives a large amount of
  % recovering
  % of lost information, it's expected to have a good effect on
  % reconstructing the matter density field from observation.
\end{section}
