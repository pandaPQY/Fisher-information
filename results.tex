\begin{section}{Conclusion and Discussion}
 \label{sec:conclusion}
    We use the code "CubeP3M" to generate 137 independent dark matter density fields, then give the reconstructed deformation potentials which are pure divergent using APM method. We analyze the power spectra of both the density fields and the deformation potentials, after which we give the cross correlation matrix. We find that the power spectra are highly correlated on small scales, since these scales are in non-linear regime. But after reconstruction, the strongly correlated regime shinks from $k\sim0.1$ to $k\sim0.5$. We also calculate the cumulative information, and find that the plateau of the reconstructed information curve in the tranlinear regime rises by a factor of 20.  

   The new reconstruction method successfully recovers the lost linear information on the mildly non-linear scale, at least twice better than previous methods \cite{bib:Mark2006,bib:Mark2009,bib:Zhang2011,bib:Yu2012,bib:Mark2014} and pushes the non-linear scale to a smaller scale in our case. The result in dark matter density fields gives a strong motivation to adapt APM method in halo fields, neutrino fields, etc, so that we have access to know more clearly about the physics in smaller scale. Some efforts were made to improve cosmology measurements to BAO scale (e.g. \cite{bib:Daniel2007,bib:Martin2015}). APM gives the reconstructed displacement given on hte Lagrangian coordinate instead of the final Eulerian coordiante. It's successful try of BAO reconstruction in 1-D cosmology \cite{bib:Zhu2016} provided an intuitive view of the algorithm to develop the BAO reconstruction in 3-D and thus push forward the BAO research.

   APM method effectively decomposes the irrotational part and the curl part of the displacement field of particles, which would be meaningful to be compared with the decomposition of displacement field given by N-body simulation \cite{bib:Yu2016}. 
\end{section}
