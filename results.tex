\begin{section}{Conclusion}
  \label{sec:conclusion}
  We study the Moving-Mesh algorithm's ability to estimate the underlying displacement 
  field and reconstruct the linear density fields using 130 cosmological $N$-body 
  simulations.  We measure the power spectra and the associated covariance of the 
  nonlinear density fields, and the reconstructed density fields.  We quantify the 
  result by (i) cross-correlating them with the linear density fields, (ii) studying 
  the $k$-mode coupling in the correlation matrix, and (iii) computing the cumulative 
  Fisher information contained in these power spectra.  We also
  compared with the $E$-mode 
  reconstruction and logarithmic density mapping Gaussianization
  techniques.
We find that Moving-Mesh method gives better results than previous works
  (e.g. \citealt{bib:Mark2009,bib:Zhang2011,bib:HarnoisD2013}), and on scales 
  relevant to the BAO, our result approaches the optimal $E$-mode reconstruction.  
  Future steps include quantifying halo Poisson noise and bias contamination 
  from realistic measurements, and determining the quantitative impact on
  BAO and RSD measurements.  

\end{section}
