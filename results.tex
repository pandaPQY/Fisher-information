\begin{section}{Conclusion and Discussion}
 \label{sec:conclusion}
    We use the code "CubeP3M" to generate 136 independent dark matter density fields, 
then give the reconstructed deformation potentials which are pure divergent using MM 
algorithm. We analyze the power spectra of both the non-linear density fields and 
reconstructed density fields, after which we give the cross correlation matrix. We find that the power 
spectra are highly correlated on small scales, since these scales are in non-linear 
regime. But after reconstruction, the strongly correlated regime shinks from $k\sim0.2$ 
to $k\sim0.6$. We also calculate the cumulative information, and find that the plateau 
of the reconstructed information curve keeps matching the Gausisan information curve until 
$k \sim 0.3$ and reach a plateau at $k \sim 0.8$, rising by a factor of 40 compared to the 
non-linear information curve.

   We argue that the reconstructed Fisher information will still increasing to a greater magnitude in smaller scale 
since the cross correlation of the reconstructed and linear power spectrum increases 
in a higher resolution analysis cite(bib:ZhuH2016).

   The new reconstruction method successfully recovers the lost linear information on 
the mildly non-linear scale, better than previous methods 
\cite{bib:Mark2006,bib:Mark2009,bib:Zhang2011,bib:Yu2012,bib:Mark2014} and pushes the 
non-linear scale to a smaller scale in our case. The result in dark matter density fields 
gives a strong motivation to adapt MM reconstruction in halo fields, neutrino fields, etc, so that 
we have access to know more clearly about the physics in smaller scale. Some efforts were 
made to improve cosmology measurements to BAO scale (e.g. \cite{bib:Daniel2007,bib:Martin2015}). 
MM reconstruction gives the reconstructed displacement given on the Lagrangian coordinates transfromed from the 
Eulerian coordiantes. It's successful try on BAO reconstruction in 1-D 
\cite{bib:Zhu2016} and 3-D cite(bib:ZhuH2016) cosmology provided an intuitive view of the algorithm 
to push forward the BAO research.

   MM reconstruction effectively decomposes the irrotational part and the curl part of the displacement 
field of particles.
On the other hand, the reconstructed displacement might be greatly different from real displacement in N-body 
simulation, since it's sensitive to the late stage shell-crossing and non-linear process so that the original 
position of some spectific particles can be replaced by each other. It's meaningful to compare the irrotational 
displacement field through MM reocnstruction and that from E-mode displacement reconstruction \cite{bib:Yu2016} 
which decomposes cleanly the irrotational and curl part of the real displacement in N-body simulation. 
Since MM reconstruction only needs the density field input and gives a latge recovering of lost information, 
it's expected to have a good effect on reconstructing the matter density field from obsurvation.
\end{section}
