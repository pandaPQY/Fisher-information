\begin{section}{Conclusion}
  \label{sec:conclusion}
  From 130 large scale structure $N$-body simulations, 
  we study the moving-mesh reconstruction technique on estimating the underlying displacement
  field and reconstruction of linear density fields in Lagrangian
  space. We measure the power spectra and their covariance of
  the original nonlinear density fields $\delta_N$, reconstructed density fields $\delta_R$,
  and quantify the result by (i) cross-correlating them respect to linear density fields $\delta_L$,
  (ii) studying the $k$-mode coupling in the covariance matrix, and (iii)
  computing the cumulative Fisher information contained
  in these power spectra. We also apply $E$-mode reconstruction
  and logarithmic transformation Gaussianization and compare
  with the moving-mesh reconstruction result.
  
  The MM reconstruction technique largely recovers the linear correlation
  (Fig. \ref{fig:cp}).  By reducing the mode-mode coupling in nonlinear
  regime (Fig. \ref{fig:corrall}), it successfully recovers the lost linear
  information on mildly nonlinear scales and increases the saturated
  information (Fig. \ref{fig:fisherinfo}).  The result is
  better than previous methods,
  e.g. \cite{bib:Mark2006,bib:Mark2009,bib:Zhang2011,bib:Yu2012}.
  The next step/work is to test the accuracy of BAO dilation scale after the MM reconstruction.

\end{section}
